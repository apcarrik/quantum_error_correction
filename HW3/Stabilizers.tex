\documentclass[11pt]{article}

    \usepackage[breakable]{tcolorbox}
    \usepackage{parskip} % Stop auto-indenting (to mimic markdown behaviour)
    
    \usepackage{iftex}
    \ifPDFTeX
    	\usepackage[T1]{fontenc}
    	\usepackage{mathpazo}
    \else
    	\usepackage{fontspec}
    \fi

    % Basic figure setup, for now with no caption control since it's done
    % automatically by Pandoc (which extracts ![](path) syntax from Markdown).
    \usepackage{graphicx}
    % Maintain compatibility with old templates. Remove in nbconvert 6.0
    \let\Oldincludegraphics\includegraphics
    % Ensure that by default, figures have no caption (until we provide a
    % proper Figure object with a Caption API and a way to capture that
    % in the conversion process - todo).
    \usepackage{caption}
    \DeclareCaptionFormat{nocaption}{}
    \captionsetup{format=nocaption,aboveskip=0pt,belowskip=0pt}

    \usepackage{float}
    \floatplacement{figure}{H} % forces figures to be placed at the correct location
    \usepackage{xcolor} % Allow colors to be defined
    \usepackage{enumerate} % Needed for markdown enumerations to work
    \usepackage{geometry} % Used to adjust the document margins
    \usepackage{amsmath} % Equations
    \usepackage{amssymb} % Equations
    \usepackage{textcomp} % defines textquotesingle
    % Hack from http://tex.stackexchange.com/a/47451/13684:
    \AtBeginDocument{%
        \def\PYZsq{\textquotesingle}% Upright quotes in Pygmentized code
    }
    \usepackage{upquote} % Upright quotes for verbatim code
    \usepackage{eurosym} % defines \euro
    \usepackage[mathletters]{ucs} % Extended unicode (utf-8) support
    \usepackage{fancyvrb} % verbatim replacement that allows latex
    \usepackage{grffile} % extends the file name processing of package graphics 
                         % to support a larger range
    \makeatletter % fix for old versions of grffile with XeLaTeX
    \@ifpackagelater{grffile}{2019/11/01}
    {
      % Do nothing on new versions
    }
    {
      \def\Gread@@xetex#1{%
        \IfFileExists{"\Gin@base".bb}%
        {\Gread@eps{\Gin@base.bb}}%
        {\Gread@@xetex@aux#1}%
      }
    }
    \makeatother
    \usepackage[Export]{adjustbox} % Used to constrain images to a maximum size
    \adjustboxset{max size={0.9\linewidth}{0.9\paperheight}}

    % The hyperref package gives us a pdf with properly built
    % internal navigation ('pdf bookmarks' for the table of contents,
    % internal cross-reference links, web links for URLs, etc.)
    \usepackage{hyperref}
    % The default LaTeX title has an obnoxious amount of whitespace. By default,
    % titling removes some of it. It also provides customization options.
    \usepackage{titling}
    \usepackage{longtable} % longtable support required by pandoc >1.10
    \usepackage{booktabs}  % table support for pandoc > 1.12.2
    \usepackage[inline]{enumitem} % IRkernel/repr support (it uses the enumerate* environment)
    \usepackage[normalem]{ulem} % ulem is needed to support strikethroughs (\sout)
                                % normalem makes italics be italics, not underlines
    \usepackage{mathrsfs}
    

    
    % Colors for the hyperref package
    \definecolor{urlcolor}{rgb}{0,.145,.698}
    \definecolor{linkcolor}{rgb}{.71,0.21,0.01}
    \definecolor{citecolor}{rgb}{.12,.54,.11}

    % ANSI colors
    \definecolor{ansi-black}{HTML}{3E424D}
    \definecolor{ansi-black-intense}{HTML}{282C36}
    \definecolor{ansi-red}{HTML}{E75C58}
    \definecolor{ansi-red-intense}{HTML}{B22B31}
    \definecolor{ansi-green}{HTML}{00A250}
    \definecolor{ansi-green-intense}{HTML}{007427}
    \definecolor{ansi-yellow}{HTML}{DDB62B}
    \definecolor{ansi-yellow-intense}{HTML}{B27D12}
    \definecolor{ansi-blue}{HTML}{208FFB}
    \definecolor{ansi-blue-intense}{HTML}{0065CA}
    \definecolor{ansi-magenta}{HTML}{D160C4}
    \definecolor{ansi-magenta-intense}{HTML}{A03196}
    \definecolor{ansi-cyan}{HTML}{60C6C8}
    \definecolor{ansi-cyan-intense}{HTML}{258F8F}
    \definecolor{ansi-white}{HTML}{C5C1B4}
    \definecolor{ansi-white-intense}{HTML}{A1A6B2}
    \definecolor{ansi-default-inverse-fg}{HTML}{FFFFFF}
    \definecolor{ansi-default-inverse-bg}{HTML}{000000}

    % common color for the border for error outputs.
    \definecolor{outerrorbackground}{HTML}{FFDFDF}

    % commands and environments needed by pandoc snippets
    % extracted from the output of `pandoc -s`
    \providecommand{\tightlist}{%
      \setlength{\itemsep}{0pt}\setlength{\parskip}{0pt}}
    \DefineVerbatimEnvironment{Highlighting}{Verbatim}{commandchars=\\\{\}}
    % Add ',fontsize=\small' for more characters per line
    \newenvironment{Shaded}{}{}
    \newcommand{\KeywordTok}[1]{\textcolor[rgb]{0.00,0.44,0.13}{\textbf{{#1}}}}
    \newcommand{\DataTypeTok}[1]{\textcolor[rgb]{0.56,0.13,0.00}{{#1}}}
    \newcommand{\DecValTok}[1]{\textcolor[rgb]{0.25,0.63,0.44}{{#1}}}
    \newcommand{\BaseNTok}[1]{\textcolor[rgb]{0.25,0.63,0.44}{{#1}}}
    \newcommand{\FloatTok}[1]{\textcolor[rgb]{0.25,0.63,0.44}{{#1}}}
    \newcommand{\CharTok}[1]{\textcolor[rgb]{0.25,0.44,0.63}{{#1}}}
    \newcommand{\StringTok}[1]{\textcolor[rgb]{0.25,0.44,0.63}{{#1}}}
    \newcommand{\CommentTok}[1]{\textcolor[rgb]{0.38,0.63,0.69}{\textit{{#1}}}}
    \newcommand{\OtherTok}[1]{\textcolor[rgb]{0.00,0.44,0.13}{{#1}}}
    \newcommand{\AlertTok}[1]{\textcolor[rgb]{1.00,0.00,0.00}{\textbf{{#1}}}}
    \newcommand{\FunctionTok}[1]{\textcolor[rgb]{0.02,0.16,0.49}{{#1}}}
    \newcommand{\RegionMarkerTok}[1]{{#1}}
    \newcommand{\ErrorTok}[1]{\textcolor[rgb]{1.00,0.00,0.00}{\textbf{{#1}}}}
    \newcommand{\NormalTok}[1]{{#1}}
    
    % Additional commands for more recent versions of Pandoc
    \newcommand{\ConstantTok}[1]{\textcolor[rgb]{0.53,0.00,0.00}{{#1}}}
    \newcommand{\SpecialCharTok}[1]{\textcolor[rgb]{0.25,0.44,0.63}{{#1}}}
    \newcommand{\VerbatimStringTok}[1]{\textcolor[rgb]{0.25,0.44,0.63}{{#1}}}
    \newcommand{\SpecialStringTok}[1]{\textcolor[rgb]{0.73,0.40,0.53}{{#1}}}
    \newcommand{\ImportTok}[1]{{#1}}
    \newcommand{\DocumentationTok}[1]{\textcolor[rgb]{0.73,0.13,0.13}{\textit{{#1}}}}
    \newcommand{\AnnotationTok}[1]{\textcolor[rgb]{0.38,0.63,0.69}{\textbf{\textit{{#1}}}}}
    \newcommand{\CommentVarTok}[1]{\textcolor[rgb]{0.38,0.63,0.69}{\textbf{\textit{{#1}}}}}
    \newcommand{\VariableTok}[1]{\textcolor[rgb]{0.10,0.09,0.49}{{#1}}}
    \newcommand{\ControlFlowTok}[1]{\textcolor[rgb]{0.00,0.44,0.13}{\textbf{{#1}}}}
    \newcommand{\OperatorTok}[1]{\textcolor[rgb]{0.40,0.40,0.40}{{#1}}}
    \newcommand{\BuiltInTok}[1]{{#1}}
    \newcommand{\ExtensionTok}[1]{{#1}}
    \newcommand{\PreprocessorTok}[1]{\textcolor[rgb]{0.74,0.48,0.00}{{#1}}}
    \newcommand{\AttributeTok}[1]{\textcolor[rgb]{0.49,0.56,0.16}{{#1}}}
    \newcommand{\InformationTok}[1]{\textcolor[rgb]{0.38,0.63,0.69}{\textbf{\textit{{#1}}}}}
    \newcommand{\WarningTok}[1]{\textcolor[rgb]{0.38,0.63,0.69}{\textbf{\textit{{#1}}}}}
    
    
    % Define a nice break command that doesn't care if a line doesn't already
    % exist.
    \def\br{\hspace*{\fill} \\* }
    % Math Jax compatibility definitions
    \def\gt{>}
    \def\lt{<}
    \let\Oldtex\TeX
    \let\Oldlatex\LaTeX
    \renewcommand{\TeX}{\textrm{\Oldtex}}
    \renewcommand{\LaTeX}{\textrm{\Oldlatex}}
    % Document parameters
    % Document title
    \title{Stabilizers}
    
    
    
    
    
% Pygments definitions
\makeatletter
\def\PY@reset{\let\PY@it=\relax \let\PY@bf=\relax%
    \let\PY@ul=\relax \let\PY@tc=\relax%
    \let\PY@bc=\relax \let\PY@ff=\relax}
\def\PY@tok#1{\csname PY@tok@#1\endcsname}
\def\PY@toks#1+{\ifx\relax#1\empty\else%
    \PY@tok{#1}\expandafter\PY@toks\fi}
\def\PY@do#1{\PY@bc{\PY@tc{\PY@ul{%
    \PY@it{\PY@bf{\PY@ff{#1}}}}}}}
\def\PY#1#2{\PY@reset\PY@toks#1+\relax+\PY@do{#2}}

\@namedef{PY@tok@w}{\def\PY@tc##1{\textcolor[rgb]{0.73,0.73,0.73}{##1}}}
\@namedef{PY@tok@c}{\let\PY@it=\textit\def\PY@tc##1{\textcolor[rgb]{0.24,0.48,0.48}{##1}}}
\@namedef{PY@tok@cp}{\def\PY@tc##1{\textcolor[rgb]{0.61,0.40,0.00}{##1}}}
\@namedef{PY@tok@k}{\let\PY@bf=\textbf\def\PY@tc##1{\textcolor[rgb]{0.00,0.50,0.00}{##1}}}
\@namedef{PY@tok@kp}{\def\PY@tc##1{\textcolor[rgb]{0.00,0.50,0.00}{##1}}}
\@namedef{PY@tok@kt}{\def\PY@tc##1{\textcolor[rgb]{0.69,0.00,0.25}{##1}}}
\@namedef{PY@tok@o}{\def\PY@tc##1{\textcolor[rgb]{0.40,0.40,0.40}{##1}}}
\@namedef{PY@tok@ow}{\let\PY@bf=\textbf\def\PY@tc##1{\textcolor[rgb]{0.67,0.13,1.00}{##1}}}
\@namedef{PY@tok@nb}{\def\PY@tc##1{\textcolor[rgb]{0.00,0.50,0.00}{##1}}}
\@namedef{PY@tok@nf}{\def\PY@tc##1{\textcolor[rgb]{0.00,0.00,1.00}{##1}}}
\@namedef{PY@tok@nc}{\let\PY@bf=\textbf\def\PY@tc##1{\textcolor[rgb]{0.00,0.00,1.00}{##1}}}
\@namedef{PY@tok@nn}{\let\PY@bf=\textbf\def\PY@tc##1{\textcolor[rgb]{0.00,0.00,1.00}{##1}}}
\@namedef{PY@tok@ne}{\let\PY@bf=\textbf\def\PY@tc##1{\textcolor[rgb]{0.80,0.25,0.22}{##1}}}
\@namedef{PY@tok@nv}{\def\PY@tc##1{\textcolor[rgb]{0.10,0.09,0.49}{##1}}}
\@namedef{PY@tok@no}{\def\PY@tc##1{\textcolor[rgb]{0.53,0.00,0.00}{##1}}}
\@namedef{PY@tok@nl}{\def\PY@tc##1{\textcolor[rgb]{0.46,0.46,0.00}{##1}}}
\@namedef{PY@tok@ni}{\let\PY@bf=\textbf\def\PY@tc##1{\textcolor[rgb]{0.44,0.44,0.44}{##1}}}
\@namedef{PY@tok@na}{\def\PY@tc##1{\textcolor[rgb]{0.41,0.47,0.13}{##1}}}
\@namedef{PY@tok@nt}{\let\PY@bf=\textbf\def\PY@tc##1{\textcolor[rgb]{0.00,0.50,0.00}{##1}}}
\@namedef{PY@tok@nd}{\def\PY@tc##1{\textcolor[rgb]{0.67,0.13,1.00}{##1}}}
\@namedef{PY@tok@s}{\def\PY@tc##1{\textcolor[rgb]{0.73,0.13,0.13}{##1}}}
\@namedef{PY@tok@sd}{\let\PY@it=\textit\def\PY@tc##1{\textcolor[rgb]{0.73,0.13,0.13}{##1}}}
\@namedef{PY@tok@si}{\let\PY@bf=\textbf\def\PY@tc##1{\textcolor[rgb]{0.64,0.35,0.47}{##1}}}
\@namedef{PY@tok@se}{\let\PY@bf=\textbf\def\PY@tc##1{\textcolor[rgb]{0.67,0.36,0.12}{##1}}}
\@namedef{PY@tok@sr}{\def\PY@tc##1{\textcolor[rgb]{0.64,0.35,0.47}{##1}}}
\@namedef{PY@tok@ss}{\def\PY@tc##1{\textcolor[rgb]{0.10,0.09,0.49}{##1}}}
\@namedef{PY@tok@sx}{\def\PY@tc##1{\textcolor[rgb]{0.00,0.50,0.00}{##1}}}
\@namedef{PY@tok@m}{\def\PY@tc##1{\textcolor[rgb]{0.40,0.40,0.40}{##1}}}
\@namedef{PY@tok@gh}{\let\PY@bf=\textbf\def\PY@tc##1{\textcolor[rgb]{0.00,0.00,0.50}{##1}}}
\@namedef{PY@tok@gu}{\let\PY@bf=\textbf\def\PY@tc##1{\textcolor[rgb]{0.50,0.00,0.50}{##1}}}
\@namedef{PY@tok@gd}{\def\PY@tc##1{\textcolor[rgb]{0.63,0.00,0.00}{##1}}}
\@namedef{PY@tok@gi}{\def\PY@tc##1{\textcolor[rgb]{0.00,0.52,0.00}{##1}}}
\@namedef{PY@tok@gr}{\def\PY@tc##1{\textcolor[rgb]{0.89,0.00,0.00}{##1}}}
\@namedef{PY@tok@ge}{\let\PY@it=\textit}
\@namedef{PY@tok@gs}{\let\PY@bf=\textbf}
\@namedef{PY@tok@gp}{\let\PY@bf=\textbf\def\PY@tc##1{\textcolor[rgb]{0.00,0.00,0.50}{##1}}}
\@namedef{PY@tok@go}{\def\PY@tc##1{\textcolor[rgb]{0.44,0.44,0.44}{##1}}}
\@namedef{PY@tok@gt}{\def\PY@tc##1{\textcolor[rgb]{0.00,0.27,0.87}{##1}}}
\@namedef{PY@tok@err}{\def\PY@bc##1{{\setlength{\fboxsep}{\string -\fboxrule}\fcolorbox[rgb]{1.00,0.00,0.00}{1,1,1}{\strut ##1}}}}
\@namedef{PY@tok@kc}{\let\PY@bf=\textbf\def\PY@tc##1{\textcolor[rgb]{0.00,0.50,0.00}{##1}}}
\@namedef{PY@tok@kd}{\let\PY@bf=\textbf\def\PY@tc##1{\textcolor[rgb]{0.00,0.50,0.00}{##1}}}
\@namedef{PY@tok@kn}{\let\PY@bf=\textbf\def\PY@tc##1{\textcolor[rgb]{0.00,0.50,0.00}{##1}}}
\@namedef{PY@tok@kr}{\let\PY@bf=\textbf\def\PY@tc##1{\textcolor[rgb]{0.00,0.50,0.00}{##1}}}
\@namedef{PY@tok@bp}{\def\PY@tc##1{\textcolor[rgb]{0.00,0.50,0.00}{##1}}}
\@namedef{PY@tok@fm}{\def\PY@tc##1{\textcolor[rgb]{0.00,0.00,1.00}{##1}}}
\@namedef{PY@tok@vc}{\def\PY@tc##1{\textcolor[rgb]{0.10,0.09,0.49}{##1}}}
\@namedef{PY@tok@vg}{\def\PY@tc##1{\textcolor[rgb]{0.10,0.09,0.49}{##1}}}
\@namedef{PY@tok@vi}{\def\PY@tc##1{\textcolor[rgb]{0.10,0.09,0.49}{##1}}}
\@namedef{PY@tok@vm}{\def\PY@tc##1{\textcolor[rgb]{0.10,0.09,0.49}{##1}}}
\@namedef{PY@tok@sa}{\def\PY@tc##1{\textcolor[rgb]{0.73,0.13,0.13}{##1}}}
\@namedef{PY@tok@sb}{\def\PY@tc##1{\textcolor[rgb]{0.73,0.13,0.13}{##1}}}
\@namedef{PY@tok@sc}{\def\PY@tc##1{\textcolor[rgb]{0.73,0.13,0.13}{##1}}}
\@namedef{PY@tok@dl}{\def\PY@tc##1{\textcolor[rgb]{0.73,0.13,0.13}{##1}}}
\@namedef{PY@tok@s2}{\def\PY@tc##1{\textcolor[rgb]{0.73,0.13,0.13}{##1}}}
\@namedef{PY@tok@sh}{\def\PY@tc##1{\textcolor[rgb]{0.73,0.13,0.13}{##1}}}
\@namedef{PY@tok@s1}{\def\PY@tc##1{\textcolor[rgb]{0.73,0.13,0.13}{##1}}}
\@namedef{PY@tok@mb}{\def\PY@tc##1{\textcolor[rgb]{0.40,0.40,0.40}{##1}}}
\@namedef{PY@tok@mf}{\def\PY@tc##1{\textcolor[rgb]{0.40,0.40,0.40}{##1}}}
\@namedef{PY@tok@mh}{\def\PY@tc##1{\textcolor[rgb]{0.40,0.40,0.40}{##1}}}
\@namedef{PY@tok@mi}{\def\PY@tc##1{\textcolor[rgb]{0.40,0.40,0.40}{##1}}}
\@namedef{PY@tok@il}{\def\PY@tc##1{\textcolor[rgb]{0.40,0.40,0.40}{##1}}}
\@namedef{PY@tok@mo}{\def\PY@tc##1{\textcolor[rgb]{0.40,0.40,0.40}{##1}}}
\@namedef{PY@tok@ch}{\let\PY@it=\textit\def\PY@tc##1{\textcolor[rgb]{0.24,0.48,0.48}{##1}}}
\@namedef{PY@tok@cm}{\let\PY@it=\textit\def\PY@tc##1{\textcolor[rgb]{0.24,0.48,0.48}{##1}}}
\@namedef{PY@tok@cpf}{\let\PY@it=\textit\def\PY@tc##1{\textcolor[rgb]{0.24,0.48,0.48}{##1}}}
\@namedef{PY@tok@c1}{\let\PY@it=\textit\def\PY@tc##1{\textcolor[rgb]{0.24,0.48,0.48}{##1}}}
\@namedef{PY@tok@cs}{\let\PY@it=\textit\def\PY@tc##1{\textcolor[rgb]{0.24,0.48,0.48}{##1}}}

\def\PYZbs{\char`\\}
\def\PYZus{\char`\_}
\def\PYZob{\char`\{}
\def\PYZcb{\char`\}}
\def\PYZca{\char`\^}
\def\PYZam{\char`\&}
\def\PYZlt{\char`\<}
\def\PYZgt{\char`\>}
\def\PYZsh{\char`\#}
\def\PYZpc{\char`\%}
\def\PYZdl{\char`\$}
\def\PYZhy{\char`\-}
\def\PYZsq{\char`\'}
\def\PYZdq{\char`\"}
\def\PYZti{\char`\~}
% for compatibility with earlier versions
\def\PYZat{@}
\def\PYZlb{[}
\def\PYZrb{]}
\makeatother


    % For linebreaks inside Verbatim environment from package fancyvrb. 
    \makeatletter
        \newbox\Wrappedcontinuationbox 
        \newbox\Wrappedvisiblespacebox 
        \newcommand*\Wrappedvisiblespace {\textcolor{red}{\textvisiblespace}} 
        \newcommand*\Wrappedcontinuationsymbol {\textcolor{red}{\llap{\tiny$\m@th\hookrightarrow$}}} 
        \newcommand*\Wrappedcontinuationindent {3ex } 
        \newcommand*\Wrappedafterbreak {\kern\Wrappedcontinuationindent\copy\Wrappedcontinuationbox} 
        % Take advantage of the already applied Pygments mark-up to insert 
        % potential linebreaks for TeX processing. 
        %        {, <, #, %, $, ' and ": go to next line. 
        %        _, }, ^, &, >, - and ~: stay at end of broken line. 
        % Use of \textquotesingle for straight quote. 
        \newcommand*\Wrappedbreaksatspecials {% 
            \def\PYGZus{\discretionary{\char`\_}{\Wrappedafterbreak}{\char`\_}}% 
            \def\PYGZob{\discretionary{}{\Wrappedafterbreak\char`\{}{\char`\{}}% 
            \def\PYGZcb{\discretionary{\char`\}}{\Wrappedafterbreak}{\char`\}}}% 
            \def\PYGZca{\discretionary{\char`\^}{\Wrappedafterbreak}{\char`\^}}% 
            \def\PYGZam{\discretionary{\char`\&}{\Wrappedafterbreak}{\char`\&}}% 
            \def\PYGZlt{\discretionary{}{\Wrappedafterbreak\char`\<}{\char`\<}}% 
            \def\PYGZgt{\discretionary{\char`\>}{\Wrappedafterbreak}{\char`\>}}% 
            \def\PYGZsh{\discretionary{}{\Wrappedafterbreak\char`\#}{\char`\#}}% 
            \def\PYGZpc{\discretionary{}{\Wrappedafterbreak\char`\%}{\char`\%}}% 
            \def\PYGZdl{\discretionary{}{\Wrappedafterbreak\char`\$}{\char`\$}}% 
            \def\PYGZhy{\discretionary{\char`\-}{\Wrappedafterbreak}{\char`\-}}% 
            \def\PYGZsq{\discretionary{}{\Wrappedafterbreak\textquotesingle}{\textquotesingle}}% 
            \def\PYGZdq{\discretionary{}{\Wrappedafterbreak\char`\"}{\char`\"}}% 
            \def\PYGZti{\discretionary{\char`\~}{\Wrappedafterbreak}{\char`\~}}% 
        } 
        % Some characters . , ; ? ! / are not pygmentized. 
        % This macro makes them "active" and they will insert potential linebreaks 
        \newcommand*\Wrappedbreaksatpunct {% 
            \lccode`\~`\.\lowercase{\def~}{\discretionary{\hbox{\char`\.}}{\Wrappedafterbreak}{\hbox{\char`\.}}}% 
            \lccode`\~`\,\lowercase{\def~}{\discretionary{\hbox{\char`\,}}{\Wrappedafterbreak}{\hbox{\char`\,}}}% 
            \lccode`\~`\;\lowercase{\def~}{\discretionary{\hbox{\char`\;}}{\Wrappedafterbreak}{\hbox{\char`\;}}}% 
            \lccode`\~`\:\lowercase{\def~}{\discretionary{\hbox{\char`\:}}{\Wrappedafterbreak}{\hbox{\char`\:}}}% 
            \lccode`\~`\?\lowercase{\def~}{\discretionary{\hbox{\char`\?}}{\Wrappedafterbreak}{\hbox{\char`\?}}}% 
            \lccode`\~`\!\lowercase{\def~}{\discretionary{\hbox{\char`\!}}{\Wrappedafterbreak}{\hbox{\char`\!}}}% 
            \lccode`\~`\/\lowercase{\def~}{\discretionary{\hbox{\char`\/}}{\Wrappedafterbreak}{\hbox{\char`\/}}}% 
            \catcode`\.\active
            \catcode`\,\active 
            \catcode`\;\active
            \catcode`\:\active
            \catcode`\?\active
            \catcode`\!\active
            \catcode`\/\active 
            \lccode`\~`\~ 	
        }
    \makeatother

    \let\OriginalVerbatim=\Verbatim
    \makeatletter
    \renewcommand{\Verbatim}[1][1]{%
        %\parskip\z@skip
        \sbox\Wrappedcontinuationbox {\Wrappedcontinuationsymbol}%
        \sbox\Wrappedvisiblespacebox {\FV@SetupFont\Wrappedvisiblespace}%
        \def\FancyVerbFormatLine ##1{\hsize\linewidth
            \vtop{\raggedright\hyphenpenalty\z@\exhyphenpenalty\z@
                \doublehyphendemerits\z@\finalhyphendemerits\z@
                \strut ##1\strut}%
        }%
        % If the linebreak is at a space, the latter will be displayed as visible
        % space at end of first line, and a continuation symbol starts next line.
        % Stretch/shrink are however usually zero for typewriter font.
        \def\FV@Space {%
            \nobreak\hskip\z@ plus\fontdimen3\font minus\fontdimen4\font
            \discretionary{\copy\Wrappedvisiblespacebox}{\Wrappedafterbreak}
            {\kern\fontdimen2\font}%
        }%
        
        % Allow breaks at special characters using \PYG... macros.
        \Wrappedbreaksatspecials
        % Breaks at punctuation characters . , ; ? ! and / need catcode=\active 	
        \OriginalVerbatim[#1,codes*=\Wrappedbreaksatpunct]%
    }
    \makeatother

    % Exact colors from NB
    \definecolor{incolor}{HTML}{303F9F}
    \definecolor{outcolor}{HTML}{D84315}
    \definecolor{cellborder}{HTML}{CFCFCF}
    \definecolor{cellbackground}{HTML}{F7F7F7}
    
    % prompt
    \makeatletter
    \newcommand{\boxspacing}{\kern\kvtcb@left@rule\kern\kvtcb@boxsep}
    \makeatother
    \newcommand{\prompt}[4]{
        {\ttfamily\llap{{\color{#2}[#3]:\hspace{3pt}#4}}\vspace{-\baselineskip}}
    }
    

    
    % Prevent overflowing lines due to hard-to-break entities
    \sloppy 
    % Setup hyperref package
    \hypersetup{
      breaklinks=true,  % so long urls are correctly broken across lines
      colorlinks=true,
      urlcolor=urlcolor,
      linkcolor=linkcolor,
      citecolor=citecolor,
      }
    % Slightly bigger margins than the latex defaults
    
    \geometry{verbose,tmargin=1in,bmargin=1in,lmargin=1in,rmargin=1in}
    
    

\begin{document}
    
    \maketitle
    
    

    
    \hypertarget{homework-3}{%
\section{Homework 3}\label{homework-3}}

\hypertarget{ece-621-spring-2022}{%
\subsection{ECE 621 (Spring 2022)}\label{ece-621-spring-2022}}

\hypertarget{adam-carriker}{%
\subsubsection{Adam Carriker}\label{adam-carriker}}

Copyright 2022, Adam Carriker. All rights reserved.

    \hypertarget{problem-1}{%
\subsection{Problem 1}\label{problem-1}}

    \hypertarget{a.}{%
\subsection{1a.}\label{a.}}

\hypertarget{show-the-state-0i-is-stabilized-by-the-group-generated-by-ziiixi-iiy}{%
\paragraph{Show the state
\textbar0\textgreater\textbar+\textgreater\textbar+i\textgreater{} is
stabilized by the group generated by \textless ZII,IXI,
IIY\textgreater{}}\label{show-the-state-0i-is-stabilized-by-the-group-generated-by-ziiixi-iiy}}

    \begin{tcolorbox}[breakable, size=fbox, boxrule=1pt, pad at break*=1mm,colback=cellbackground, colframe=cellborder]
\prompt{In}{incolor}{1}{\boxspacing}
\begin{Verbatim}[commandchars=\\\{\}]
\PY{k+kn}{import} \PY{n+nn}{numpy} \PY{k}{as} \PY{n+nn}{np}
\PY{k+kn}{import} \PY{n+nn}{cmath}

\PY{n}{ZERO} \PY{o}{=} \PY{n}{np}\PY{o}{.}\PY{n}{array}\PY{p}{(}\PY{p}{[}\PY{p}{[}\PY{l+m+mi}{0}\PY{p}{,}\PY{l+m+mi}{0}\PY{p}{]}\PY{p}{,}\PY{p}{[}\PY{l+m+mi}{0}\PY{p}{,}\PY{l+m+mi}{0}\PY{p}{]}\PY{p}{]}\PY{p}{,} \PY{n}{dtype}\PY{o}{=}\PY{n+nb}{complex}\PY{p}{)}
\PY{n}{I}\PY{o}{=}\PY{n}{np}\PY{o}{.}\PY{n}{array}\PY{p}{(}\PY{p}{[}\PY{p}{[}\PY{l+m+mi}{1}\PY{p}{,}\PY{l+m+mi}{0}\PY{p}{]}\PY{p}{,}\PY{p}{[}\PY{l+m+mi}{0}\PY{p}{,}\PY{l+m+mi}{1}\PY{p}{]}\PY{p}{]}\PY{p}{,} \PY{n}{dtype}\PY{o}{=}\PY{n+nb}{complex}\PY{p}{)}
\PY{n}{X}\PY{o}{=}\PY{n}{np}\PY{o}{.}\PY{n}{array}\PY{p}{(}\PY{p}{[}\PY{p}{[}\PY{l+m+mi}{0}\PY{p}{,}\PY{l+m+mi}{1}\PY{p}{]}\PY{p}{,} \PY{p}{[}\PY{l+m+mi}{1}\PY{p}{,}\PY{l+m+mi}{0}\PY{p}{]}\PY{p}{]}\PY{p}{,} \PY{n}{dtype} \PY{o}{=} \PY{n+nb}{complex}\PY{p}{)}
\PY{n}{Y}\PY{o}{=}\PY{n}{np}\PY{o}{.}\PY{n}{array}\PY{p}{(}\PY{p}{[}\PY{p}{[}\PY{l+m+mi}{0}\PY{p}{,}\PY{o}{\PYZhy{}}\PY{l+m+mi}{1}\PY{n}{j}\PY{p}{]}\PY{p}{,} \PY{p}{[}\PY{l+m+mi}{1}\PY{n}{j}\PY{p}{,}\PY{l+m+mi}{0}\PY{p}{]}\PY{p}{]}\PY{p}{,} \PY{n}{dtype} \PY{o}{=} \PY{n+nb}{complex}\PY{p}{)}
\PY{n}{Z}\PY{o}{=}\PY{n}{np}\PY{o}{.}\PY{n}{array}\PY{p}{(}\PY{p}{[}\PY{p}{[}\PY{l+m+mi}{1}\PY{p}{,}\PY{l+m+mi}{0}\PY{p}{]}\PY{p}{,} \PY{p}{[}\PY{l+m+mi}{0}\PY{p}{,}\PY{o}{\PYZhy{}}\PY{l+m+mi}{1}\PY{p}{]}\PY{p}{]}\PY{p}{,} \PY{n}{dtype} \PY{o}{=} \PY{n+nb}{complex}\PY{p}{)}
\PY{n}{iI}\PY{o}{=}\PY{n}{np}\PY{o}{.}\PY{n}{dot}\PY{p}{(}\PY{l+m+mi}{1}\PY{n}{j}\PY{p}{,}\PY{n}{I}\PY{p}{)}
\PY{n}{iX}\PY{o}{=}\PY{n}{np}\PY{o}{.}\PY{n}{dot}\PY{p}{(}\PY{l+m+mi}{1}\PY{n}{j}\PY{p}{,}\PY{n}{X}\PY{p}{)}
\PY{n}{iY}\PY{o}{=}\PY{n}{np}\PY{o}{.}\PY{n}{dot}\PY{p}{(}\PY{l+m+mi}{1}\PY{n}{j}\PY{p}{,}\PY{n}{Y}\PY{p}{)}
\PY{n}{iZ}\PY{o}{=}\PY{n}{np}\PY{o}{.}\PY{n}{dot}\PY{p}{(}\PY{l+m+mi}{1}\PY{n}{j}\PY{p}{,}\PY{n}{Z}\PY{p}{)}

\PY{n}{k\PYZus{}0} \PY{o}{=} \PY{n}{np}\PY{o}{.}\PY{n}{array}\PY{p}{(}\PY{p}{[}\PY{p}{[}\PY{l+m+mi}{1}\PY{p}{]}\PY{p}{,}\PY{p}{[}\PY{l+m+mi}{0}\PY{p}{]}\PY{p}{]}\PY{p}{)}
\PY{n}{k\PYZus{}1} \PY{o}{=} \PY{n}{np}\PY{o}{.}\PY{n}{array}\PY{p}{(}\PY{p}{[}\PY{p}{[}\PY{l+m+mi}{0}\PY{p}{]}\PY{p}{,}\PY{p}{[}\PY{l+m+mi}{1}\PY{p}{]}\PY{p}{]}\PY{p}{)}
\PY{n}{k\PYZus{}p} \PY{o}{=} \PY{l+m+mi}{1}\PY{o}{/}\PY{n}{cmath}\PY{o}{.}\PY{n}{sqrt}\PY{p}{(}\PY{l+m+mi}{2}\PY{p}{)}\PY{o}{*}\PY{p}{(}\PY{n}{k\PYZus{}0} \PY{o}{+} \PY{n}{k\PYZus{}1}\PY{p}{)}
\PY{n}{k\PYZus{}m} \PY{o}{=} \PY{l+m+mi}{1}\PY{o}{/}\PY{n}{cmath}\PY{o}{.}\PY{n}{sqrt}\PY{p}{(}\PY{l+m+mi}{2}\PY{p}{)}\PY{o}{*}\PY{p}{(}\PY{n}{k\PYZus{}0} \PY{o}{\PYZhy{}} \PY{n}{k\PYZus{}1}\PY{p}{)}
\PY{n}{k\PYZus{}pi} \PY{o}{=} \PY{l+m+mi}{1}\PY{o}{/}\PY{n}{cmath}\PY{o}{.}\PY{n}{sqrt}\PY{p}{(}\PY{l+m+mi}{2}\PY{p}{)}\PY{o}{*}\PY{p}{(}\PY{n}{k\PYZus{}0} \PY{o}{+} \PY{l+m+mi}{1}\PY{n}{j}\PY{o}{*}\PY{n}{k\PYZus{}1}\PY{p}{)}
\PY{n}{k\PYZus{}mi} \PY{o}{=} \PY{l+m+mi}{1}\PY{o}{/}\PY{n}{cmath}\PY{o}{.}\PY{n}{sqrt}\PY{p}{(}\PY{l+m+mi}{2}\PY{p}{)}\PY{o}{*}\PY{p}{(}\PY{n}{k\PYZus{}0} \PY{o}{\PYZhy{}} \PY{l+m+mi}{1}\PY{n}{j}\PY{o}{*}\PY{n}{k\PYZus{}1}\PY{p}{)}

\PY{n}{init\PYZus{}state} \PY{o}{=} \PY{n}{np}\PY{o}{.}\PY{n}{kron}\PY{p}{(}\PY{n}{k\PYZus{}0}\PY{p}{,} \PY{n}{np}\PY{o}{.}\PY{n}{kron}\PY{p}{(}\PY{n}{k\PYZus{}p}\PY{p}{,} \PY{n}{k\PYZus{}pi}\PY{p}{)}\PY{p}{)}
\PY{n+nb}{print}\PY{p}{(}\PY{l+s+s2}{\PYZdq{}}\PY{l+s+s2}{Initial state |ψ\PYZgt{} = |0\PYZgt{}|+\PYZgt{}|+i\PYZgt{} = }\PY{l+s+se}{\PYZbs{}n}\PY{l+s+s2}{\PYZdq{}}\PY{p}{,} \PY{n}{init\PYZus{}state}\PY{p}{)}

\PY{n}{group} \PY{o}{=} \PY{p}{\PYZob{}}
    \PY{l+s+s2}{\PYZdq{}}\PY{l+s+s2}{ZII}\PY{l+s+s2}{\PYZdq{}} \PY{p}{:} \PY{n}{np}\PY{o}{.}\PY{n}{kron}\PY{p}{(}\PY{n}{Z}\PY{p}{,}\PY{n}{np}\PY{o}{.}\PY{n}{kron}\PY{p}{(}\PY{n}{I}\PY{p}{,}\PY{n}{I}\PY{p}{)}\PY{p}{)}\PY{p}{,}
    \PY{l+s+s2}{\PYZdq{}}\PY{l+s+s2}{IXI}\PY{l+s+s2}{\PYZdq{}} \PY{p}{:} \PY{n}{np}\PY{o}{.}\PY{n}{kron}\PY{p}{(}\PY{n}{I}\PY{p}{,}\PY{n}{np}\PY{o}{.}\PY{n}{kron}\PY{p}{(}\PY{n}{X}\PY{p}{,}\PY{n}{I}\PY{p}{)}\PY{p}{)}\PY{p}{,}
    \PY{l+s+s2}{\PYZdq{}}\PY{l+s+s2}{IIY}\PY{l+s+s2}{\PYZdq{}} \PY{p}{:} \PY{n}{np}\PY{o}{.}\PY{n}{kron}\PY{p}{(}\PY{n}{I}\PY{p}{,}\PY{n}{np}\PY{o}{.}\PY{n}{kron}\PY{p}{(}\PY{n}{I}\PY{p}{,}\PY{n}{Y}\PY{p}{)}\PY{p}{)}
\PY{p}{\PYZcb{}}
\PY{n}{group\PYZus{}items} \PY{o}{=} \PY{n+nb}{list}\PY{p}{(}\PY{n}{group}\PY{o}{.}\PY{n}{items}\PY{p}{(}\PY{p}{)}\PY{p}{)}

\PY{n+nb}{print}\PY{p}{(}\PY{l+s+s2}{\PYZdq{}}\PY{l+s+s2}{group \PYZlt{}ZII, IXI, IIY\PYZgt{} :}\PY{l+s+s2}{\PYZdq{}}\PY{p}{)}
\PY{k}{for} \PY{p}{(}\PY{n}{key}\PY{p}{,}\PY{n}{val}\PY{p}{)} \PY{o+ow}{in} \PY{n}{group\PYZus{}items}\PY{p}{:}
    \PY{n+nb}{print}\PY{p}{(}\PY{n}{key}\PY{p}{,} \PY{l+s+s2}{\PYZdq{}}\PY{l+s+s2}{=}\PY{l+s+s2}{\PYZdq{}}\PY{p}{,} \PY{n}{val}\PY{p}{)}
      
\PY{k}{for} \PY{p}{(}\PY{n}{key}\PY{p}{,}\PY{n}{val}\PY{p}{)} \PY{o+ow}{in} \PY{n}{group\PYZus{}items}\PY{p}{:}
    \PY{n}{op}\PY{o}{=}\PY{n}{np}\PY{o}{.}\PY{n}{matmul}\PY{p}{(}\PY{n}{val}\PY{p}{,}\PY{n}{init\PYZus{}state}\PY{p}{)}
    \PY{n+nb}{print}\PY{p}{(}\PY{l+s+sa}{f}\PY{l+s+s2}{\PYZdq{}}\PY{l+s+si}{\PYZob{}}\PY{n}{key}\PY{l+s+si}{\PYZcb{}}\PY{l+s+s2}{|ψ\PYZgt{} = }\PY{l+s+s2}{\PYZdq{}}\PY{p}{,} \PY{n}{op}\PY{p}{)}
    \PY{n+nb}{print}\PY{p}{(}\PY{l+s+s2}{\PYZdq{}}\PY{l+s+s2}{= |ψ\PYZgt{} ✅}\PY{l+s+s2}{\PYZdq{}} \PY{k}{if} \PY{n}{np}\PY{o}{.}\PY{n}{array\PYZus{}equal}\PY{p}{(}\PY{n}{op}\PY{p}{,}\PY{n}{init\PYZus{}state}\PY{p}{)} \PY{k}{else} \PY{l+s+s2}{\PYZdq{}}\PY{l+s+s2}{≠ |ψ\PYZgt{} ❌}\PY{l+s+s2}{\PYZdq{}}\PY{p}{)}
\end{Verbatim}
\end{tcolorbox}

    \begin{Verbatim}[commandchars=\\\{\}]
Initial state |ψ> = |0>|+>|+i> =
 [[0.5+0.j ]
 [0. +0.5j]
 [0.5+0.j ]
 [0. +0.5j]
 [0. +0.j ]
 [0. +0.j ]
 [0. +0.j ]
 [0. +0.j ]]
group <ZII, IXI, IIY> :
ZII = [[ 1.+0.j  0.+0.j  0.+0.j  0.+0.j  0.+0.j  0.+0.j  0.+0.j  0.+0.j]
 [ 0.+0.j  1.+0.j  0.+0.j  0.+0.j  0.+0.j  0.+0.j  0.+0.j  0.+0.j]
 [ 0.+0.j  0.+0.j  1.+0.j  0.+0.j  0.+0.j  0.+0.j  0.+0.j  0.+0.j]
 [ 0.+0.j  0.+0.j  0.+0.j  1.+0.j  0.+0.j  0.+0.j  0.+0.j  0.+0.j]
 [ 0.+0.j  0.+0.j  0.+0.j  0.+0.j -1.+0.j -0.+0.j -0.+0.j -0.+0.j]
 [ 0.+0.j  0.+0.j  0.+0.j  0.+0.j -0.+0.j -1.+0.j -0.+0.j -0.+0.j]
 [ 0.+0.j  0.+0.j  0.+0.j  0.+0.j -0.+0.j -0.+0.j -1.+0.j -0.+0.j]
 [ 0.+0.j  0.+0.j  0.+0.j  0.+0.j -0.+0.j -0.+0.j -0.+0.j -1.+0.j]]
IXI = [[0.+0.j 0.+0.j 1.+0.j 0.+0.j 0.+0.j 0.+0.j 0.+0.j 0.+0.j]
 [0.+0.j 0.+0.j 0.+0.j 1.+0.j 0.+0.j 0.+0.j 0.+0.j 0.+0.j]
 [1.+0.j 0.+0.j 0.+0.j 0.+0.j 0.+0.j 0.+0.j 0.+0.j 0.+0.j]
 [0.+0.j 1.+0.j 0.+0.j 0.+0.j 0.+0.j 0.+0.j 0.+0.j 0.+0.j]
 [0.+0.j 0.+0.j 0.+0.j 0.+0.j 0.+0.j 0.+0.j 1.+0.j 0.+0.j]
 [0.+0.j 0.+0.j 0.+0.j 0.+0.j 0.+0.j 0.+0.j 0.+0.j 1.+0.j]
 [0.+0.j 0.+0.j 0.+0.j 0.+0.j 1.+0.j 0.+0.j 0.+0.j 0.+0.j]
 [0.+0.j 0.+0.j 0.+0.j 0.+0.j 0.+0.j 1.+0.j 0.+0.j 0.+0.j]]
IIY = [[0.+0.j 0.-1.j 0.+0.j 0.+0.j 0.+0.j 0.+0.j 0.+0.j 0.+0.j]
 [0.+1.j 0.+0.j 0.+0.j 0.+0.j 0.+0.j 0.+0.j 0.+0.j 0.+0.j]
 [0.+0.j 0.+0.j 0.+0.j 0.-1.j 0.+0.j 0.+0.j 0.+0.j 0.+0.j]
 [0.+0.j 0.+0.j 0.+1.j 0.+0.j 0.+0.j 0.+0.j 0.+0.j 0.+0.j]
 [0.+0.j 0.+0.j 0.+0.j 0.+0.j 0.+0.j 0.-1.j 0.+0.j 0.+0.j]
 [0.+0.j 0.+0.j 0.+0.j 0.+0.j 0.+1.j 0.+0.j 0.+0.j 0.+0.j]
 [0.+0.j 0.+0.j 0.+0.j 0.+0.j 0.+0.j 0.+0.j 0.+0.j 0.-1.j]
 [0.+0.j 0.+0.j 0.+0.j 0.+0.j 0.+0.j 0.+0.j 0.+1.j 0.+0.j]]
ZII|ψ> =  [[0.5+0.j ]
 [0. +0.5j]
 [0.5+0.j ]
 [0. +0.5j]
 [0. +0.j ]
 [0. +0.j ]
 [0. +0.j ]
 [0. +0.j ]]
= |ψ> ✅
IXI|ψ> =  [[0.5+0.j ]
 [0. +0.5j]
 [0.5+0.j ]
 [0. +0.5j]
 [0. +0.j ]
 [0. +0.j ]
 [0. +0.j ]
 [0. +0.j ]]
= |ψ> ✅
IIY|ψ> =  [[0.5+0.j ]
 [0. +0.5j]
 [0.5+0.j ]
 [0. +0.5j]
 [0. +0.j ]
 [0. +0.j ]
 [0. +0.j ]
 [0. +0.j ]]
= |ψ> ✅
    \end{Verbatim}

    Since each generator operator has \textbar ψ\textgreater{} as an
eigenstate with eigenvalue +1, \textbar ψ\textgreater{} is stabilized by
this group.

    \hypertarget{b.}{%
\subsection{1b.}\label{b.}}

\hypertarget{write-down-an-alternative-set-of-generators.}{%
\paragraph{Write down an alternative set of
generators.}\label{write-down-an-alternative-set-of-generators.}}

    Idea: try every permutation of Pauli gates \{I,X,Y,Z\} in a 3-deep gate

Better idea: use the fact that Clifford operators map Pauli operators to
Pauli operators, which means they also must map stabilizer states to
stabilizer states.

    \begin{tcolorbox}[breakable, size=fbox, boxrule=1pt, pad at break*=1mm,colback=cellbackground, colframe=cellborder]
\prompt{In}{incolor}{2}{\boxspacing}
\begin{Verbatim}[commandchars=\\\{\}]
\PY{k}{def} \PY{n+nf}{aboutequal}\PY{p}{(}\PY{n}{a}\PY{p}{,}\PY{n}{b}\PY{p}{,}\PY{n}{rt}\PY{o}{=}\PY{l+m+mf}{1e\PYZhy{}15}\PY{p}{,} \PY{n}{at}\PY{o}{=}\PY{l+m+mf}{1e\PYZhy{}15}\PY{p}{)}\PY{p}{:}
    \PY{k}{return} \PY{k+kc}{True} \PY{k}{if} \PY{n}{np}\PY{o}{.}\PY{n}{array\PYZus{}equal}\PY{p}{(}\PY{n}{np}\PY{o}{.}\PY{n}{allclose}\PY{p}{(}\PY{n}{a}\PY{p}{,}\PY{n}{b}\PY{p}{,}\PY{n}{rtol}\PY{o}{=}\PY{n}{rt}\PY{p}{,} \PY{n}{atol}\PY{o}{=}\PY{n}{at}\PY{p}{)}\PY{p}{,}\PY{n}{np}\PY{o}{.}\PY{n}{array}\PY{p}{(}\PY{p}{[}\PY{p}{[}\PY{k+kc}{True}\PY{p}{,}\PY{k+kc}{True}\PY{p}{]}\PY{p}{,}\PY{p}{[}\PY{k+kc}{True}\PY{p}{,}\PY{k+kc}{True}\PY{p}{]}\PY{p}{]}\PY{p}{)}\PY{p}{)}\PYZbs{}
    \PY{k}{else} \PY{k+kc}{False}
\end{Verbatim}
\end{tcolorbox}

    \begin{tcolorbox}[breakable, size=fbox, boxrule=1pt, pad at break*=1mm,colback=cellbackground, colframe=cellborder]
\prompt{In}{incolor}{3}{\boxspacing}
\begin{Verbatim}[commandchars=\\\{\}]
\PY{c+c1}{\PYZsh{} proof below:}

\PY{n}{H} \PY{o}{=} \PY{n}{cmath}\PY{o}{.}\PY{n}{sqrt}\PY{p}{(}\PY{l+m+mi}{1}\PY{o}{/}\PY{l+m+mi}{2}\PY{p}{)}\PY{o}{*}\PY{p}{(}\PY{n}{X}\PY{o}{+}\PY{n}{Z}\PY{p}{)}
\PY{n}{S} \PY{o}{=} \PY{n}{cmath}\PY{o}{.}\PY{n}{exp}\PY{p}{(}\PY{l+m+mi}{1}\PY{n}{j}\PY{o}{*}\PY{n}{cmath}\PY{o}{.}\PY{n}{pi}\PY{o}{/}\PY{l+m+mi}{4}\PY{p}{)}\PY{o}{*}\PY{n}{cmath}\PY{o}{.}\PY{n}{sqrt}\PY{p}{(}\PY{l+m+mi}{1}\PY{o}{/}\PY{l+m+mi}{2}\PY{p}{)}\PY{o}{*}\PY{p}{(}\PY{n}{I}\PY{o}{\PYZhy{}}\PY{n}{iZ}\PY{p}{)}
\PY{n}{CNOT} \PY{o}{=} \PY{l+m+mi}{1}\PY{o}{/}\PY{l+m+mi}{2}\PY{o}{*}\PY{p}{(}\PY{n}{np}\PY{o}{.}\PY{n}{kron}\PY{p}{(}\PY{n}{I}\PY{p}{,}\PY{n}{I}\PY{p}{)}\PY{o}{+}\PY{n}{np}\PY{o}{.}\PY{n}{kron}\PY{p}{(}\PY{n}{Z}\PY{p}{,}\PY{n}{I}\PY{p}{)}\PY{o}{+}\PY{n}{np}\PY{o}{.}\PY{n}{kron}\PY{p}{(}\PY{n}{I}\PY{p}{,}\PY{n}{X}\PY{p}{)}\PY{o}{\PYZhy{}}\PY{n}{np}\PY{o}{.}\PY{n}{kron}\PY{p}{(}\PY{n}{Z}\PY{p}{,}\PY{n}{X}\PY{p}{)}\PY{p}{)}

\PY{n}{computedZ} \PY{o}{=} \PY{n}{np}\PY{o}{.}\PY{n}{matmul}\PY{p}{(}\PY{n}{H}\PY{p}{,}\PY{n}{np}\PY{o}{.}\PY{n}{matmul}\PY{p}{(}\PY{n}{X}\PY{p}{,}\PY{n}{H}\PY{p}{)}\PY{p}{)} 
\PY{n}{computedX} \PY{o}{=} \PY{n}{np}\PY{o}{.}\PY{n}{matmul}\PY{p}{(}\PY{n}{H}\PY{p}{,}\PY{n}{np}\PY{o}{.}\PY{n}{matmul}\PY{p}{(}\PY{n}{Z}\PY{p}{,}\PY{n}{H}\PY{p}{)}\PY{p}{)} 
\PY{n}{computedI} \PY{o}{=} \PY{n}{np}\PY{o}{.}\PY{n}{matmul}\PY{p}{(}\PY{n}{H}\PY{p}{,}\PY{n}{np}\PY{o}{.}\PY{n}{matmul}\PY{p}{(}\PY{n}{I}\PY{p}{,}\PY{n}{H}\PY{p}{)}\PY{p}{)}
\PY{n}{computedY} \PY{o}{=} \PY{n}{np}\PY{o}{.}\PY{n}{matmul}\PY{p}{(}\PY{n}{H}\PY{p}{,}\PY{n}{np}\PY{o}{.}\PY{n}{matmul}\PY{p}{(}\PY{n}{Y}\PY{p}{,}\PY{n}{H}\PY{p}{)}\PY{p}{)}

\PY{n+nb}{print}\PY{p}{(}\PY{l+s+s2}{\PYZdq{}}\PY{l+s+s2}{HXH = Z? :}\PY{l+s+s2}{\PYZdq{}}\PY{p}{,} \PY{n}{aboutequal}\PY{p}{(}\PY{n}{Z}\PY{p}{,}\PY{n}{computedZ}\PY{p}{)}\PY{p}{)}
\PY{n+nb}{print}\PY{p}{(}\PY{l+s+s2}{\PYZdq{}}\PY{l+s+s2}{HXH = X? :}\PY{l+s+s2}{\PYZdq{}}\PY{p}{,} \PY{n}{aboutequal}\PY{p}{(}\PY{n}{X}\PY{p}{,}\PY{n}{computedX}\PY{p}{)}\PY{p}{)}
\PY{n+nb}{print}\PY{p}{(}\PY{l+s+s2}{\PYZdq{}}\PY{l+s+s2}{HIH = I? :}\PY{l+s+s2}{\PYZdq{}}\PY{p}{,} \PY{n}{aboutequal}\PY{p}{(}\PY{n}{I}\PY{p}{,}\PY{n}{computedI}\PY{p}{)}\PY{p}{)}
\PY{n+nb}{print}\PY{p}{(}\PY{l+s+s2}{\PYZdq{}}\PY{l+s+s2}{computedY = }\PY{l+s+s2}{\PYZdq{}}\PY{p}{,} \PY{n}{computedY}\PY{p}{)}
\PY{n+nb}{print}\PY{p}{(}\PY{l+s+s2}{\PYZdq{}}\PY{l+s+s2}{HYH = Y? :}\PY{l+s+s2}{\PYZdq{}}\PY{p}{,} \PY{n}{aboutequal}\PY{p}{(}\PY{n}{Y}\PY{p}{,}\PY{n}{computedY}\PY{p}{)}\PY{p}{)}
\PY{n+nb}{print}\PY{p}{(}\PY{l+s+s2}{\PYZdq{}}\PY{l+s+s2}{HYH = \PYZhy{}Y? :}\PY{l+s+s2}{\PYZdq{}}\PY{p}{,} \PY{n}{aboutequal}\PY{p}{(}\PY{o}{\PYZhy{}}\PY{n}{Y}\PY{p}{,}\PY{n}{computedY}\PY{p}{)}\PY{p}{)}
\end{Verbatim}
\end{tcolorbox}

    \begin{Verbatim}[commandchars=\\\{\}]
HXH = Z? : False
HXH = X? : False
HIH = I? : False
computedY =  [[0.-4.26642159e-17j 0.+1.00000000e+00j]
 [0.-1.00000000e+00j 0.+4.26642159e-17j]]
HYH = Y? : False
HYH = -Y? : False
    \end{Verbatim}

    So, now we can try applying the H operator before and after each of the
gates in the previous group to get the new group. For example:

\begin{verbatim}
ZII -> (HZH)(HIH)(HIH) = (X)(I)(I) = XII
IXI -> (HIH)(HXH)(HIH) = (I)(Z)(I) = IZI
IIY -> (HIH)(HIH)(HYH) = (I)(I)(-Y) = II(-Y)
\end{verbatim}

    \begin{tcolorbox}[breakable, size=fbox, boxrule=1pt, pad at break*=1mm,colback=cellbackground, colframe=cellborder]
\prompt{In}{incolor}{4}{\boxspacing}
\begin{Verbatim}[commandchars=\\\{\}]
\PY{n}{group2} \PY{o}{=} \PY{p}{\PYZob{}}
    \PY{l+s+s2}{\PYZdq{}}\PY{l+s+s2}{ZII}\PY{l+s+s2}{\PYZdq{}} \PY{p}{:} \PY{n}{np}\PY{o}{.}\PY{n}{kron}\PY{p}{(}\PY{n}{X}\PY{p}{,}\PY{n}{np}\PY{o}{.}\PY{n}{kron}\PY{p}{(}\PY{n}{I}\PY{p}{,}\PY{n}{I}\PY{p}{)}\PY{p}{)}\PY{p}{,}
    \PY{l+s+s2}{\PYZdq{}}\PY{l+s+s2}{IXI}\PY{l+s+s2}{\PYZdq{}} \PY{p}{:} \PY{n}{np}\PY{o}{.}\PY{n}{kron}\PY{p}{(}\PY{n}{I}\PY{p}{,}\PY{n}{np}\PY{o}{.}\PY{n}{kron}\PY{p}{(}\PY{n}{Z}\PY{p}{,}\PY{n}{I}\PY{p}{)}\PY{p}{)}\PY{p}{,}
    \PY{l+s+s2}{\PYZdq{}}\PY{l+s+s2}{IInY}\PY{l+s+s2}{\PYZdq{}} \PY{p}{:} \PY{n}{np}\PY{o}{.}\PY{n}{kron}\PY{p}{(}\PY{n}{I}\PY{p}{,}\PY{n}{np}\PY{o}{.}\PY{n}{kron}\PY{p}{(}\PY{n}{I}\PY{p}{,}\PY{o}{\PYZhy{}}\PY{n}{Y}\PY{p}{)}\PY{p}{)}
\PY{p}{\PYZcb{}}
\PY{n}{group2\PYZus{}items} \PY{o}{=} \PY{n+nb}{list}\PY{p}{(}\PY{n}{group}\PY{o}{.}\PY{n}{items}\PY{p}{(}\PY{p}{)}\PY{p}{)}

\PY{n+nb}{print}\PY{p}{(}\PY{l+s+s2}{\PYZdq{}}\PY{l+s+s2}{alternate group \PYZlt{}XII, IZI, II(\PYZhy{}Y)\PYZgt{} :}\PY{l+s+s2}{\PYZdq{}}\PY{p}{)}
\PY{k}{for} \PY{p}{(}\PY{n}{key}\PY{p}{,}\PY{n}{val}\PY{p}{)} \PY{o+ow}{in} \PY{n}{group2\PYZus{}items}\PY{p}{:}
    \PY{n+nb}{print}\PY{p}{(}\PY{n}{key}\PY{p}{,} \PY{l+s+s2}{\PYZdq{}}\PY{l+s+s2}{=}\PY{l+s+s2}{\PYZdq{}}\PY{p}{,} \PY{n}{val}\PY{p}{)}
      
\PY{k}{for} \PY{p}{(}\PY{n}{key}\PY{p}{,}\PY{n}{val}\PY{p}{)} \PY{o+ow}{in} \PY{n}{group2\PYZus{}items}\PY{p}{:}
    \PY{n}{op}\PY{o}{=}\PY{n}{np}\PY{o}{.}\PY{n}{matmul}\PY{p}{(}\PY{n}{val}\PY{p}{,}\PY{n}{init\PYZus{}state}\PY{p}{)}
    \PY{n+nb}{print}\PY{p}{(}\PY{l+s+sa}{f}\PY{l+s+s2}{\PYZdq{}}\PY{l+s+si}{\PYZob{}}\PY{n}{key}\PY{l+s+si}{\PYZcb{}}\PY{l+s+s2}{|ψ\PYZgt{} = }\PY{l+s+s2}{\PYZdq{}}\PY{p}{,} \PY{n}{op}\PY{p}{)}
    \PY{n+nb}{print}\PY{p}{(}\PY{l+s+s2}{\PYZdq{}}\PY{l+s+s2}{= |ψ\PYZgt{} ✅}\PY{l+s+s2}{\PYZdq{}} \PY{k}{if} \PY{n}{np}\PY{o}{.}\PY{n}{array\PYZus{}equal}\PY{p}{(}\PY{n}{op}\PY{p}{,}\PY{n}{init\PYZus{}state}\PY{p}{)} \PY{k}{else} \PY{l+s+s2}{\PYZdq{}}\PY{l+s+s2}{≠ |ψ\PYZgt{} ❌}\PY{l+s+s2}{\PYZdq{}}\PY{p}{)}
\end{Verbatim}
\end{tcolorbox}

    \begin{Verbatim}[commandchars=\\\{\}]
alternate group <XII, IZI, II(-Y)> :
ZII = [[ 1.+0.j  0.+0.j  0.+0.j  0.+0.j  0.+0.j  0.+0.j  0.+0.j  0.+0.j]
 [ 0.+0.j  1.+0.j  0.+0.j  0.+0.j  0.+0.j  0.+0.j  0.+0.j  0.+0.j]
 [ 0.+0.j  0.+0.j  1.+0.j  0.+0.j  0.+0.j  0.+0.j  0.+0.j  0.+0.j]
 [ 0.+0.j  0.+0.j  0.+0.j  1.+0.j  0.+0.j  0.+0.j  0.+0.j  0.+0.j]
 [ 0.+0.j  0.+0.j  0.+0.j  0.+0.j -1.+0.j -0.+0.j -0.+0.j -0.+0.j]
 [ 0.+0.j  0.+0.j  0.+0.j  0.+0.j -0.+0.j -1.+0.j -0.+0.j -0.+0.j]
 [ 0.+0.j  0.+0.j  0.+0.j  0.+0.j -0.+0.j -0.+0.j -1.+0.j -0.+0.j]
 [ 0.+0.j  0.+0.j  0.+0.j  0.+0.j -0.+0.j -0.+0.j -0.+0.j -1.+0.j]]
IXI = [[0.+0.j 0.+0.j 1.+0.j 0.+0.j 0.+0.j 0.+0.j 0.+0.j 0.+0.j]
 [0.+0.j 0.+0.j 0.+0.j 1.+0.j 0.+0.j 0.+0.j 0.+0.j 0.+0.j]
 [1.+0.j 0.+0.j 0.+0.j 0.+0.j 0.+0.j 0.+0.j 0.+0.j 0.+0.j]
 [0.+0.j 1.+0.j 0.+0.j 0.+0.j 0.+0.j 0.+0.j 0.+0.j 0.+0.j]
 [0.+0.j 0.+0.j 0.+0.j 0.+0.j 0.+0.j 0.+0.j 1.+0.j 0.+0.j]
 [0.+0.j 0.+0.j 0.+0.j 0.+0.j 0.+0.j 0.+0.j 0.+0.j 1.+0.j]
 [0.+0.j 0.+0.j 0.+0.j 0.+0.j 1.+0.j 0.+0.j 0.+0.j 0.+0.j]
 [0.+0.j 0.+0.j 0.+0.j 0.+0.j 0.+0.j 1.+0.j 0.+0.j 0.+0.j]]
IIY = [[0.+0.j 0.-1.j 0.+0.j 0.+0.j 0.+0.j 0.+0.j 0.+0.j 0.+0.j]
 [0.+1.j 0.+0.j 0.+0.j 0.+0.j 0.+0.j 0.+0.j 0.+0.j 0.+0.j]
 [0.+0.j 0.+0.j 0.+0.j 0.-1.j 0.+0.j 0.+0.j 0.+0.j 0.+0.j]
 [0.+0.j 0.+0.j 0.+1.j 0.+0.j 0.+0.j 0.+0.j 0.+0.j 0.+0.j]
 [0.+0.j 0.+0.j 0.+0.j 0.+0.j 0.+0.j 0.-1.j 0.+0.j 0.+0.j]
 [0.+0.j 0.+0.j 0.+0.j 0.+0.j 0.+1.j 0.+0.j 0.+0.j 0.+0.j]
 [0.+0.j 0.+0.j 0.+0.j 0.+0.j 0.+0.j 0.+0.j 0.+0.j 0.-1.j]
 [0.+0.j 0.+0.j 0.+0.j 0.+0.j 0.+0.j 0.+0.j 0.+1.j 0.+0.j]]
ZII|ψ> =  [[0.5+0.j ]
 [0. +0.5j]
 [0.5+0.j ]
 [0. +0.5j]
 [0. +0.j ]
 [0. +0.j ]
 [0. +0.j ]
 [0. +0.j ]]
= |ψ> ✅
IXI|ψ> =  [[0.5+0.j ]
 [0. +0.5j]
 [0.5+0.j ]
 [0. +0.5j]
 [0. +0.j ]
 [0. +0.j ]
 [0. +0.j ]
 [0. +0.j ]]
= |ψ> ✅
IIY|ψ> =  [[0.5+0.j ]
 [0. +0.5j]
 [0.5+0.j ]
 [0. +0.5j]
 [0. +0.j ]
 [0. +0.j ]
 [0. +0.j ]
 [0. +0.j ]]
= |ψ> ✅
    \end{Verbatim}

    Therefore, the group \textless XII,IZI,II(-Y)\textgreater{} is also a
generator for the state
\textbar0\textgreater\textbar+\textgreater\textbar+i\textgreater.

    \hypertarget{c}{%
\subsection{1c)}\label{c}}

\hypertarget{find-a-gate-sequence-composed-of-cnot-h-and-s00i11-you-can-write-the-circuit-if-you-wish-that-transforms-the-state-above-into-a-state-stabilized-by-xxxzziizz.}{%
\paragraph{Find a gate sequence composed of CNOT, H, and
S=\textbar0\textgreater\textless0\textbar+i\textbar1\textgreater\textless1\textbar,
(you can write the circuit, if you wish), that transforms the state
above into a state stabilized by
\textless XXX,ZZI,IZZ\textgreater.}\label{find-a-gate-sequence-composed-of-cnot-h-and-s00i11-you-can-write-the-circuit-if-you-wish-that-transforms-the-state-above-into-a-state-stabilized-by-xxxzziizz.}}

    \begin{tcolorbox}[breakable, size=fbox, boxrule=1pt, pad at break*=1mm,colback=cellbackground, colframe=cellborder]
\prompt{In}{incolor}{5}{\boxspacing}
\begin{Verbatim}[commandchars=\\\{\}]
\PY{n}{cnot12} \PY{o}{=} \PY{n}{np}\PY{o}{.}\PY{n}{kron}\PY{p}{(}\PY{n}{CNOT}\PY{p}{,} \PY{n}{I}\PY{p}{)}
\PY{n+nb}{print}\PY{p}{(}\PY{n}{cnot12}\PY{p}{)}
\PY{n}{cnot23} \PY{o}{=} \PY{n}{np}\PY{o}{.}\PY{n}{kron}\PY{p}{(} \PY{n}{I}\PY{p}{,} \PY{n}{CNOT}\PY{p}{)}
\PY{n+nb}{print}\PY{p}{(}\PY{n}{cnot23}\PY{p}{)}

\PY{n}{state1} \PY{o}{=} \PY{n}{np}\PY{o}{.}\PY{n}{kron}\PY{p}{(}\PY{n}{k\PYZus{}p}\PY{p}{,}\PY{n}{np}\PY{o}{.}\PY{n}{kron}\PY{p}{(}\PY{n}{k\PYZus{}p}\PY{p}{,}\PY{n}{k\PYZus{}0}\PY{p}{)}\PY{p}{)}
\PY{n+nb}{print}\PY{p}{(}\PY{n}{state1}\PY{p}{)}

\PY{n+nb}{print}\PY{p}{(}\PY{n}{np}\PY{o}{.}\PY{n}{matmul}\PY{p}{(}\PY{n}{cnot12}\PY{p}{,}\PY{n}{state1}\PY{p}{)}\PY{p}{)}
\end{Verbatim}
\end{tcolorbox}

    \begin{Verbatim}[commandchars=\\\{\}]
[[1.+0.j 0.+0.j 0.+0.j 0.+0.j 0.+0.j 0.+0.j 0.+0.j 0.+0.j]
 [0.+0.j 1.+0.j 0.+0.j 0.+0.j 0.+0.j 0.+0.j 0.+0.j 0.+0.j]
 [0.+0.j 0.+0.j 1.+0.j 0.+0.j 0.+0.j 0.+0.j 0.+0.j 0.+0.j]
 [0.+0.j 0.+0.j 0.+0.j 1.+0.j 0.+0.j 0.+0.j 0.+0.j 0.+0.j]
 [0.+0.j 0.+0.j 0.+0.j 0.+0.j 0.+0.j 0.+0.j 1.+0.j 0.+0.j]
 [0.+0.j 0.+0.j 0.+0.j 0.+0.j 0.+0.j 0.+0.j 0.+0.j 1.+0.j]
 [0.+0.j 0.+0.j 0.+0.j 0.+0.j 1.+0.j 0.+0.j 0.+0.j 0.+0.j]
 [0.+0.j 0.+0.j 0.+0.j 0.+0.j 0.+0.j 1.+0.j 0.+0.j 0.+0.j]]
[[1.+0.j 0.+0.j 0.+0.j 0.+0.j 0.+0.j 0.+0.j 0.+0.j 0.+0.j]
 [0.+0.j 1.+0.j 0.+0.j 0.+0.j 0.+0.j 0.+0.j 0.+0.j 0.+0.j]
 [0.+0.j 0.+0.j 0.+0.j 1.+0.j 0.+0.j 0.+0.j 0.+0.j 0.+0.j]
 [0.+0.j 0.+0.j 1.+0.j 0.+0.j 0.+0.j 0.+0.j 0.+0.j 0.+0.j]
 [0.+0.j 0.+0.j 0.+0.j 0.+0.j 1.+0.j 0.+0.j 0.+0.j 0.+0.j]
 [0.+0.j 0.+0.j 0.+0.j 0.+0.j 0.+0.j 1.+0.j 0.+0.j 0.+0.j]
 [0.+0.j 0.+0.j 0.+0.j 0.+0.j 0.+0.j 0.+0.j 0.+0.j 1.+0.j]
 [0.+0.j 0.+0.j 0.+0.j 0.+0.j 0.+0.j 0.+0.j 1.+0.j 0.+0.j]]
[[0.5+0.j]
 [0. +0.j]
 [0.5+0.j]
 [0. +0.j]
 [0.5+0.j]
 [0. +0.j]
 [0.5+0.j]
 [0. +0.j]]
[[0.5+0.j]
 [0. +0.j]
 [0.5+0.j]
 [0. +0.j]
 [0.5+0.j]
 [0. +0.j]
 [0.5+0.j]
 [0. +0.j]]
    \end{Verbatim}

    \begin{enumerate}
\def\labelenumi{\arabic{enumi}.}
\tightlist
\item
  Starting with: generators \textless ZII,IXI,IIY\textgreater{} and
  state
  \textbar0\textgreater\textbar+\textgreater\textbar+i\textgreater.
\item
  Apply H\_1: generators H\_1\textless ZII,IXI,IIY\textgreater{}
  -\textgreater{} \textless XII,IXI,IIY\textgreater; state
  H\_1\textbar0\textgreater\textbar+\textgreater\textbar+i\textgreater{}
  -\textgreater{}
  \textbar+\textgreater\textbar+\textgreater\textbar+i\textgreater.
\item
  Apply H\_2: generators H\_2\textless XII,IXI,IIY\textgreater{}
  -\textgreater{} \textless XII,IZI,IIY\textgreater; state
  H\_2\textbar+\textgreater\textbar+\textgreater\textbar+i\textgreater{}
  -\textgreater{}
  \textbar+\textgreater\textbar0\textgreater\textbar+i\textgreater.
\item
  Apply S\_3: generators S\_3\textless XII,IZI,IIY\textgreater{}
  -\textgreater{} \textless XII,IZI,II(-X)\textgreater; state
  S\_3\textbar+\textgreater\textbar0\textgreater\textbar+i\textgreater{}
  -\textgreater{}
  \textbar+\textgreater\textbar0\textgreater\textbar-\textgreater.
\item
  Apply H\_3: generators H\_3\textless XII,IZI,II(-X)\textgreater{}
  -\textgreater{} \textless XII,IZI,II(-Z)\textgreater; state
  H\_3\textbar+\textgreater\textbar0\textgreater\textbar-\textgreater{}
  -\textgreater{}
  \textbar+\textgreater\textbar+\textgreater\textbar1\textgreater.
\item
  Apply X\_3: generators X\_3\textless XII,IZI,II(-Z)\textgreater{}
  -\textgreater{} \textless XII,IZI,IIZ\textgreater; state
  X\_3\textbar+\textgreater\textbar+\textgreater\textbar1\textgreater{}
  -\textgreater{}
  \textbar+\textgreater\textbar+\textgreater\textbar0\textgreater.
\item
  Apply CNOT\_\{1,2\}: generators
  CNOT\_\{1,2\}\textless XII,IZI,IIZ\textgreater{} -\textgreater{}
  \textless XXI,ZZI,IIZ\textgreater; state
  CNOT\_\{1,2\}\textbar+\textgreater\textbar+\textgreater\textbar0\textgreater{}
  -\textgreater{}
  \textbar+\textgreater\textbar+\textgreater\textbar0\textgreater.
\item
  Apply CNOT\_\{2,3\}: generators
  CNOT\_\{2,3\}\textless XXI,ZZI,IIZ\textgreater{} -\textgreater{}
  \textless XXX,ZZI,IZZ\textgreater; state
  CNOT\_\{2,3\}\textbar+\textgreater\textbar+\textgreater\textbar0\textgreater{}
  -\textgreater{}
  1/Sqrt{[}2{]}*(\textbar000\textgreater+\textbar110\textgreater).
\end{enumerate}

The final gate sequence to obtain generators
\textless XXX,ZZI,IZZ\textgreater{} is: CNOT\_\{2,3\} CNOT\_\{1,2\} X\_3
H\_3 S\_3 H\_2 H\_1
\textbar0\textgreater\textbar+\textgreater\textbar+i\textgreater{}

    \hypertarget{d-write-the-wavefunction-of-the-state-above.}{%
\subsection{1d) Write the wavefunction of the state
above.}\label{d-write-the-wavefunction-of-the-state-above.}}

    The wavefunction of the state above becomes
1/Sqrt{[}2{]}*(\textbar000\textgreater+\textbar110\textgreater).

    \hypertarget{problem-2}{%
\subsection{Problem 2}\label{problem-2}}

Your friend Andrew Cross tells you that he has a code on 7 qubits that
is stabilized by the following generators

\textless Z1Z5, Z2Z5, Z3Z6, Z4Z7, X3X4Y6Y7, X1X2X3Z4X5X6\textgreater{}

or

\textless ZIIIZII, IZIIZII, IIZIIZI, IIIZIIZ, IIXXIYY,
XXXZXXI\textgreater{}

    \hypertarget{show-that-this-code-can-correct-any-single-qubit-error.}{%
\subsection{1) Show that this code can correct any single qubit
error.}\label{show-that-this-code-can-correct-any-single-qubit-error.}}

    Considering only one single qubit error, we can write the quantum error
correction condition as:

\textless j\textbar\_L E\^{}\{\dagger\} \textbar k\textgreater\_l

Considering that E\^{}\{\dagger\} can be any Pauli, want we want is for
the error condition to go to 0 when we apply a stabilizer generator:

\textless{}\phi\^{}'\textbar\_L E\^{}\{\dagger\} S\_n
\textbar{}\phi\textgreater{} = 0

This will be true if a stabilizer generator has an anti-commuting gate
with respect to the error on the same qubit.

    \begin{tcolorbox}[breakable, size=fbox, boxrule=1pt, pad at break*=1mm,colback=cellbackground, colframe=cellborder]
\prompt{In}{incolor}{6}{\boxspacing}
\begin{Verbatim}[commandchars=\\\{\}]
\PY{c+c1}{\PYZsh{} Find which Paulis anti\PYZhy{}commute}


\PY{k}{def} \PY{n+nf}{find\PYZus{}anticommutator}\PY{p}{(}\PY{n}{a}\PY{p}{,}\PY{n}{b}\PY{p}{)}\PY{p}{:}
    \PY{k}{return} \PY{n}{np}\PY{o}{.}\PY{n}{matmul}\PY{p}{(}\PY{n}{a}\PY{p}{,}\PY{n}{b}\PY{p}{)}\PY{o}{+}\PY{n}{np}\PY{o}{.}\PY{n}{matmul}\PY{p}{(}\PY{n}{b}\PY{p}{,}\PY{n}{a}\PY{p}{)}

\PY{n}{pauli\PYZus{}group} \PY{o}{=} \PY{p}{\PYZob{}}
    \PY{l+s+s2}{\PYZdq{}}\PY{l+s+s2}{I}\PY{l+s+s2}{\PYZdq{}}\PY{p}{:} \PY{n}{I}\PY{p}{,}
    \PY{l+s+s2}{\PYZdq{}}\PY{l+s+s2}{X}\PY{l+s+s2}{\PYZdq{}}\PY{p}{:} \PY{n}{X}\PY{p}{,}
    \PY{l+s+s2}{\PYZdq{}}\PY{l+s+s2}{Y}\PY{l+s+s2}{\PYZdq{}}\PY{p}{:} \PY{n}{Y}\PY{p}{,}
    \PY{l+s+s2}{\PYZdq{}}\PY{l+s+s2}{Z}\PY{l+s+s2}{\PYZdq{}}\PY{p}{:} \PY{n}{Z}
\PY{p}{\PYZcb{}}

\PY{k}{for} \PY{p}{(}\PY{n}{kop1}\PY{p}{,} \PY{n}{op1}\PY{p}{)} \PY{o+ow}{in} \PY{n}{pauli\PYZus{}group}\PY{o}{.}\PY{n}{items}\PY{p}{(}\PY{p}{)}\PY{p}{:}
    \PY{k}{for} \PY{p}{(}\PY{n}{kop2}\PY{p}{,} \PY{n}{op2}\PY{p}{)} \PY{o+ow}{in} \PY{n}{pauli\PYZus{}group}\PY{o}{.}\PY{n}{items}\PY{p}{(}\PY{p}{)}\PY{p}{:}
        \PY{n}{anticommutator} \PY{o}{=} \PY{n}{find\PYZus{}anticommutator}\PY{p}{(}\PY{n}{op1}\PY{p}{,}\PY{n}{op2}\PY{p}{)}
        \PY{k}{if} \PY{n}{np}\PY{o}{.}\PY{n}{array\PYZus{}equal}\PY{p}{(}\PY{n}{anticommutator}\PY{p}{,}\PY{n}{ZERO}\PY{p}{)} \PY{o+ow}{is} \PY{k+kc}{True}\PY{p}{:} 
            \PY{n+nb}{print}\PY{p}{(}\PY{l+s+sa}{f}\PY{l+s+s2}{\PYZdq{}}\PY{l+s+s2}{Pauli operators: }\PY{l+s+si}{\PYZob{}}\PY{n}{kop1}\PY{l+s+si}{\PYZcb{}}\PY{l+s+s2}{, }\PY{l+s+si}{\PYZob{}}\PY{n}{kop2}\PY{l+s+si}{\PYZcb{}}\PY{l+s+s2}{ anti\PYZhy{}commute.}\PY{l+s+s2}{\PYZdq{}}\PY{p}{)}
\end{Verbatim}
\end{tcolorbox}

    \begin{Verbatim}[commandchars=\\\{\}]
Pauli operators: X, Y anti-commute.
Pauli operators: X, Z anti-commute.
Pauli operators: Y, X anti-commute.
Pauli operators: Y, Z anti-commute.
Pauli operators: Z, X anti-commute.
Pauli operators: Z, Y anti-commute.
    \end{Verbatim}

    Above shows that Pauli X anti-commutes with Y and Z, Y anti-commutes
with X and Z, and Z anti-commutes with X and Y. Therefore, to show that
any 1-qubit error can be corrected by our stabilizer code, we just need
to show that we can arbitrarily apply an X or Z operator at any qubit
position by selecting the correct stabilizer row S\_n.~The first four
rows of the code, \{ZIIIZII, IZIIZII, IIZIIZI, IIIZIIZ\}, ensure that a
Z gate can be applied at any qubit position. The last two rows,
\{IIXXIYY, XXXZXXI\} allow an X gate to be applied to any qubit
position. This allows us to correct any single qubit error.

    \hypertarget{determine-logical-z-and-x-operators-for-this-code.}{%
\subsection{2) Determine logical Z and X operators for this
code.}\label{determine-logical-z-and-x-operators-for-this-code.}}

    \begin{tcolorbox}[breakable, size=fbox, boxrule=1pt, pad at break*=1mm,colback=cellbackground, colframe=cellborder]
\prompt{In}{incolor}{27}{\boxspacing}
\begin{Verbatim}[commandchars=\\\{\}]
\PY{k}{def} \PY{n+nf}{find\PYZus{}commutator}\PY{p}{(}\PY{n}{a}\PY{p}{,}\PY{n}{b}\PY{p}{)}\PY{p}{:}
    \PY{k}{return} \PY{n}{np}\PY{o}{.}\PY{n}{matmul}\PY{p}{(}\PY{n}{a}\PY{p}{,}\PY{n}{b}\PY{p}{)}\PY{o}{\PYZhy{}}\PY{n}{np}\PY{o}{.}\PY{n}{matmul}\PY{p}{(}\PY{n}{b}\PY{p}{,}\PY{n}{a}\PY{p}{)}

\PY{k}{def} \PY{n+nf}{find\PYZus{}anticommutator}\PY{p}{(}\PY{n}{a}\PY{p}{,}\PY{n}{b}\PY{p}{)}\PY{p}{:}
    \PY{k}{return} \PY{n}{np}\PY{o}{.}\PY{n}{matmul}\PY{p}{(}\PY{n}{a}\PY{p}{,}\PY{n}{b}\PY{p}{)}\PY{o}{+}\PY{n}{np}\PY{o}{.}\PY{n}{matmul}\PY{p}{(}\PY{n}{b}\PY{p}{,}\PY{n}{a}\PY{p}{)}

\PY{k}{def} \PY{n+nf}{create\PYZus{}operator\PYZus{}from\PYZus{}string}\PY{p}{(}\PY{n}{op\PYZus{}string}\PY{p}{)}\PY{p}{:}
    \PY{n}{op} \PY{o}{=} \PY{n}{I}
    \PY{k}{for} \PY{n}{c} \PY{o+ow}{in} \PY{n}{op\PYZus{}string}\PY{p}{:}
        \PY{n}{op} \PY{o}{=} \PY{n}{np}\PY{o}{.}\PY{n}{matmul}\PY{p}{(}\PY{n}{op}\PY{p}{,} \PY{n}{pauli\PYZus{}group}\PY{p}{[}\PY{n}{c}\PY{p}{]}\PY{p}{)}
    \PY{k}{return} \PY{n}{op}

\PY{k}{def} \PY{n+nf}{create\PYZus{}operator\PYZus{}matrix}\PY{p}{(}\PY{n}{operator\PYZus{}string\PYZus{}dict}\PY{p}{)}\PY{p}{:}
    \PY{n}{generated\PYZus{}op} \PY{o}{=} \PY{p}{\PYZob{}}\PY{p}{\PYZcb{}}
    \PY{k}{for} \PY{p}{(}\PY{n}{k}\PY{p}{,}\PY{n}{v}\PY{p}{)} \PY{o+ow}{in} \PY{n}{operator\PYZus{}string\PYZus{}dict}\PY{o}{.}\PY{n}{items}\PY{p}{(}\PY{p}{)}\PY{p}{:}        
        \PY{n}{op} \PY{o}{=} \PY{n}{create\PYZus{}operator\PYZus{}from\PYZus{}string}\PY{p}{(}\PY{n}{v}\PY{p}{)}
        \PY{n}{generated\PYZus{}op}\PY{p}{[}\PY{n}{k}\PY{p}{]} \PY{o}{=} \PY{n}{op}
    \PY{k}{return} \PY{n}{generated\PYZus{}op}
\end{Verbatim}
\end{tcolorbox}

    \begin{tcolorbox}[breakable, size=fbox, boxrule=1pt, pad at break*=1mm,colback=cellbackground, colframe=cellborder]
\prompt{In}{incolor}{30}{\boxspacing}
\begin{Verbatim}[commandchars=\\\{\}]
\PY{c+c1}{\PYZsh{} Based off of code from: https://www.geeksforgeeks.org/print\PYZhy{}all\PYZhy{}combinations\PYZhy{}of\PYZhy{}given\PYZhy{}length/}
\PY{c+c1}{\PYZsh{} !!!! This runs in s\PYZca{}k time, very inefficient !!!!}
\PY{k}{def} \PY{n+nf}{allKLength}\PY{p}{(}\PY{n}{s}\PY{p}{,} \PY{n}{k}\PY{p}{)}\PY{p}{:} 
    \PY{n}{n} \PY{o}{=} \PY{n+nb}{len}\PY{p}{(}\PY{n}{s}\PY{p}{)}
    \PY{k}{return}\PY{p}{(}\PY{n}{allKLengthRec}\PY{p}{(}\PY{n}{s}\PY{p}{,} \PY{l+s+s2}{\PYZdq{}}\PY{l+s+s2}{\PYZdq{}}\PY{p}{,} \PY{n}{n}\PY{p}{,} \PY{n}{k}\PY{p}{,} \PY{n}{ret}\PY{o}{=}\PY{p}{[}\PY{p}{]}\PY{p}{)}\PY{p}{)}

\PY{k}{def} \PY{n+nf}{allKLengthRec}\PY{p}{(}\PY{n}{s}\PY{p}{,} \PY{n}{prefix}\PY{p}{,} \PY{n}{n}\PY{p}{,} \PY{n}{k}\PY{p}{,} \PY{n}{ret}\PY{p}{)}\PY{p}{:}
    \PY{k}{if} \PY{p}{(}\PY{n}{k} \PY{o}{==} \PY{l+m+mi}{0}\PY{p}{)} \PY{p}{:} 
        \PY{n}{ret}\PY{o}{.}\PY{n}{append}\PY{p}{(}\PY{n}{prefix}\PY{p}{)}
        \PY{k}{return} \PY{n}{ret} 
    \PY{k}{for} \PY{n}{i} \PY{o+ow}{in} \PY{n+nb}{range}\PY{p}{(}\PY{n}{n}\PY{p}{)}\PY{p}{:}
        \PY{n}{newPrefix} \PY{o}{=} \PY{n}{prefix} \PY{o}{+} \PY{n}{s}\PY{p}{[}\PY{n}{i}\PY{p}{]}
        \PY{n}{ret}\PY{o}{=}\PY{n}{allKLengthRec}\PY{p}{(}\PY{n}{s}\PY{p}{,} \PY{n}{newPrefix}\PY{p}{,} \PY{n}{n}\PY{p}{,} \PY{n}{k} \PY{o}{\PYZhy{}} \PY{l+m+mi}{1}\PY{p}{,} \PY{n}{ret}\PY{p}{)}
    \PY{k}{return} \PY{n}{ret}
\end{Verbatim}
\end{tcolorbox}

    \begin{tcolorbox}[breakable, size=fbox, boxrule=1pt, pad at break*=1mm,colback=cellbackground, colframe=cellborder]
\prompt{In}{incolor}{34}{\boxspacing}
\begin{Verbatim}[commandchars=\\\{\}]
\PY{c+c1}{\PYZsh{} brute force guess logical X and Z operators that work for code}
\PY{c+c1}{\PYZsh{} !!! This is incredibly inefficient, i\PYZsq{}m so sorry if you actually run this !!!}
\PY{k}{def} \PY{n+nf}{find\PYZus{}logical\PYZus{}ops\PYZus{}handler}\PY{p}{(}\PY{n}{generators\PYZus{}strs}\PY{p}{)}\PY{p}{:}
    \PY{p}{(}\PY{n}{lX}\PY{p}{,} \PY{n}{lZ}\PY{p}{)} \PY{o}{=} \PY{n}{find\PYZus{}logical\PYZus{}ops}\PY{p}{(}\PY{n}{generators\PYZus{}strs}\PY{p}{)}
    \PY{n}{generators\PYZus{}strs}\PY{o}{.}\PY{n}{update}\PY{p}{(}\PY{p}{\PYZob{}}
        \PY{l+s+s2}{\PYZdq{}}\PY{l+s+s2}{lX}\PY{l+s+s2}{\PYZdq{}}\PY{p}{:} \PY{n}{lX}\PY{p}{,}
        \PY{l+s+s2}{\PYZdq{}}\PY{l+s+s2}{lZ}\PY{l+s+s2}{\PYZdq{}}\PY{p}{:} \PY{n}{lZ}
    \PY{p}{\PYZcb{}}\PY{p}{)}
    \PY{n}{new\PYZus{}code\PYZus{}generated} \PY{o}{=} \PY{n}{create\PYZus{}operator\PYZus{}matrix}\PY{p}{(}\PY{n}{generators\PYZus{}strs}\PY{p}{)}

    \PY{k}{for} \PY{p}{(}\PY{n}{kop1}\PY{p}{,} \PY{n}{op1}\PY{p}{)} \PY{o+ow}{in} \PY{n}{new\PYZus{}code\PYZus{}generated}\PY{o}{.}\PY{n}{items}\PY{p}{(}\PY{p}{)}\PY{p}{:}
        \PY{k}{if}\PY{p}{(}\PY{n}{kop1} \PY{o+ow}{in} \PY{p}{(}\PY{l+s+s2}{\PYZdq{}}\PY{l+s+s2}{lZ}\PY{l+s+s2}{\PYZdq{}}\PY{p}{,} \PY{l+s+s2}{\PYZdq{}}\PY{l+s+s2}{lX}\PY{l+s+s2}{\PYZdq{}}\PY{p}{)}\PY{p}{)}\PY{p}{:}
            \PY{k}{for} \PY{p}{(}\PY{n}{kop2}\PY{p}{,} \PY{n}{op2}\PY{p}{)} \PY{o+ow}{in} \PY{n}{new\PYZus{}code\PYZus{}generated}\PY{o}{.}\PY{n}{items}\PY{p}{(}\PY{p}{)}\PY{p}{:}
                \PY{n}{commutator} \PY{o}{=} \PY{n}{find\PYZus{}commutator}\PY{p}{(}\PY{n}{op1}\PY{p}{,}\PY{n}{op2}\PY{p}{)}
                \PY{n}{anticommutator} \PY{o}{=} \PY{n}{find\PYZus{}anticommutator}\PY{p}{(}\PY{n}{op1}\PY{p}{,}\PY{n}{op2}\PY{p}{)}
                \PY{k}{if} \PY{n}{np}\PY{o}{.}\PY{n}{array\PYZus{}equal}\PY{p}{(}\PY{n}{commutator}\PY{p}{,}\PY{n}{ZERO}\PY{p}{)} \PY{o+ow}{is} \PY{k+kc}{True}\PY{p}{:} 
                    \PY{n+nb}{print}\PY{p}{(}\PY{l+s+sa}{f}\PY{l+s+s2}{\PYZdq{}}\PY{l+s+s2}{Pauli operators: }\PY{l+s+si}{\PYZob{}}\PY{n}{kop1}\PY{l+s+si}{\PYZcb{}}\PY{l+s+s2}{, }\PY{l+s+si}{\PYZob{}}\PY{n}{kop2}\PY{l+s+si}{\PYZcb{}}\PY{l+s+s2}{ commute.}\PY{l+s+s2}{\PYZdq{}}\PY{p}{)}
                \PY{k}{if} \PY{n}{np}\PY{o}{.}\PY{n}{array\PYZus{}equal}\PY{p}{(}\PY{n}{anticommutator}\PY{p}{,}\PY{n}{ZERO}\PY{p}{)} \PY{o+ow}{is} \PY{k+kc}{True}\PY{p}{:} 
                    \PY{n+nb}{print}\PY{p}{(}\PY{l+s+sa}{f}\PY{l+s+s2}{\PYZdq{}}\PY{l+s+s2}{Pauli operators: }\PY{l+s+si}{\PYZob{}}\PY{n}{kop1}\PY{l+s+si}{\PYZcb{}}\PY{l+s+s2}{, }\PY{l+s+si}{\PYZob{}}\PY{n}{kop2}\PY{l+s+si}{\PYZcb{}}\PY{l+s+s2}{ anti\PYZhy{}commute.}\PY{l+s+s2}{\PYZdq{}}\PY{p}{)}

\PY{k}{def} \PY{n+nf}{find\PYZus{}logical\PYZus{}ops}\PY{p}{(}\PY{n}{generators\PYZus{}strs}\PY{p}{)}\PY{p}{:}
    \PY{n}{n}\PY{o}{=}\PY{n+nb}{len}\PY{p}{(}\PY{n}{generators\PYZus{}strs}\PY{p}{[}\PY{l+s+s2}{\PYZdq{}}\PY{l+s+s2}{g1}\PY{l+s+s2}{\PYZdq{}}\PY{p}{]}\PY{p}{)}
    \PY{n}{generators} \PY{o}{=} \PY{n}{create\PYZus{}operator\PYZus{}matrix}\PY{p}{(}\PY{n}{generators\PYZus{}strs}\PY{p}{)}
    \PY{n}{possible\PYZus{}operators} \PY{o}{=} \PY{n}{allKLength}\PY{p}{(}\PY{n+nb}{list}\PY{p}{(}\PY{n}{pauli\PYZus{}group}\PY{o}{.}\PY{n}{keys}\PY{p}{(}\PY{p}{)}\PY{p}{)}\PY{p}{,} \PY{n}{n}\PY{p}{)}
    \PY{k}{for} \PY{n}{lZ} \PY{o+ow}{in} \PY{n}{possible\PYZus{}operators}\PY{p}{:}
        \PY{k}{for} \PY{n}{lX} \PY{o+ow}{in} \PY{n}{possible\PYZus{}operators}\PY{p}{:}
            \PY{n}{lX\PYZus{}op} \PY{o}{=} \PY{n}{create\PYZus{}operator\PYZus{}from\PYZus{}string}\PY{p}{(}\PY{n}{lX}\PY{p}{)}
            \PY{n}{lZ\PYZus{}op} \PY{o}{=} \PY{n}{create\PYZus{}operator\PYZus{}from\PYZus{}string}\PY{p}{(}\PY{n}{lZ}\PY{p}{)}
            \PY{k}{if}\PY{p}{(}\PY{n}{np}\PY{o}{.}\PY{n}{array\PYZus{}equal}\PY{p}{(}\PY{n}{find\PYZus{}anticommutator}\PY{p}{(}\PY{n}{lX\PYZus{}op}\PY{p}{,}\PY{n}{lZ\PYZus{}op}\PY{p}{)}\PY{p}{,}\PY{n}{ZERO}\PY{p}{)}\PY{p}{)}\PY{p}{:}
                \PY{n}{commute\PYZus{}flag} \PY{o}{=} \PY{k+kc}{True}
                \PY{k}{for} \PY{p}{(}\PY{n}{l\PYZus{}kop}\PY{p}{,} \PY{n}{l\PYZus{}op}\PY{p}{)} \PY{o+ow}{in} \PY{p}{(}\PY{p}{(}\PY{n}{lX}\PY{p}{,} \PY{n}{lX\PYZus{}op}\PY{p}{)}\PY{p}{,}\PY{p}{(}\PY{n}{lZ}\PY{p}{,} \PY{n}{lZ\PYZus{}op}\PY{p}{)}\PY{p}{)}\PY{p}{:}
                    \PY{k}{for} \PY{p}{(}\PY{n}{kop}\PY{p}{,} \PY{n}{op}\PY{p}{)} \PY{o+ow}{in} \PY{n}{generators}\PY{o}{.}\PY{n}{items}\PY{p}{(}\PY{p}{)}\PY{p}{:}
                        \PY{n}{commutator} \PY{o}{=} \PY{n}{find\PYZus{}commutator}\PY{p}{(}\PY{n}{l\PYZus{}op}\PY{p}{,}\PY{n}{op}\PY{p}{)}
                        \PY{n}{anticommutator} \PY{o}{=} \PY{n}{find\PYZus{}anticommutator}\PY{p}{(}\PY{n}{l\PYZus{}op}\PY{p}{,}\PY{n}{op}\PY{p}{)}
                        \PY{k}{if} \PY{n}{np}\PY{o}{.}\PY{n}{array\PYZus{}equal}\PY{p}{(}\PY{n}{commutator}\PY{p}{,}\PY{n}{ZERO}\PY{p}{)} \PY{o+ow}{is} \PY{k+kc}{False}\PY{p}{:} 
                            \PY{n}{commute\PYZus{}flag} \PY{o}{=} \PY{k+kc}{False}
                            \PY{k}{break}
                    \PY{k}{if} \PY{n}{commute\PYZus{}flag} \PY{o}{==} \PY{k+kc}{False}\PY{p}{:}
                        \PY{k}{break}
                \PY{k}{if} \PY{n}{commute\PYZus{}flag} \PY{o+ow}{is} \PY{k+kc}{True}\PY{p}{:}
                    \PY{n+nb}{print}\PY{p}{(}\PY{l+s+sa}{f}\PY{l+s+s2}{\PYZdq{}}\PY{l+s+s2}{logical operators X\PYZus{}L=}\PY{l+s+si}{\PYZob{}}\PY{n}{lX}\PY{l+s+si}{\PYZcb{}}\PY{l+s+s2}{ and Z\PYZus{}L=}\PY{l+s+si}{\PYZob{}}\PY{n}{lZ}\PY{l+s+si}{\PYZcb{}}\PY{l+s+s2}{ work.}\PY{l+s+s2}{\PYZdq{}}\PY{p}{)}
                    \PY{k}{return} \PY{p}{(}\PY{n}{lX}\PY{p}{,}\PY{n}{lZ}\PY{p}{)}
    

\PY{c+c1}{\PYZsh{} Test find\PYZus{}logical\PYZus{}ops\PYZus{}handler}
\PY{n}{five\PYZus{}qubit\PYZus{}gens\PYZus{}strs} \PY{o}{=} \PY{p}{\PYZob{}}
    \PY{l+s+s2}{\PYZdq{}}\PY{l+s+s2}{g1}\PY{l+s+s2}{\PYZdq{}}\PY{p}{:} \PY{l+s+s2}{\PYZdq{}}\PY{l+s+s2}{XZZXI}\PY{l+s+s2}{\PYZdq{}}\PY{p}{,}
    \PY{l+s+s2}{\PYZdq{}}\PY{l+s+s2}{g2}\PY{l+s+s2}{\PYZdq{}}\PY{p}{:} \PY{l+s+s2}{\PYZdq{}}\PY{l+s+s2}{IXZZX}\PY{l+s+s2}{\PYZdq{}}\PY{p}{,}
    \PY{l+s+s2}{\PYZdq{}}\PY{l+s+s2}{g3}\PY{l+s+s2}{\PYZdq{}}\PY{p}{:} \PY{l+s+s2}{\PYZdq{}}\PY{l+s+s2}{XIXZZ}\PY{l+s+s2}{\PYZdq{}}\PY{p}{,}
    \PY{l+s+s2}{\PYZdq{}}\PY{l+s+s2}{g4}\PY{l+s+s2}{\PYZdq{}}\PY{p}{:} \PY{l+s+s2}{\PYZdq{}}\PY{l+s+s2}{ZXIXZ}\PY{l+s+s2}{\PYZdq{}}\PY{p}{,}
\PY{p}{\PYZcb{}}

\PY{n}{find\PYZus{}logical\PYZus{}ops\PYZus{}handler}\PY{p}{(}\PY{n}{five\PYZus{}qubit\PYZus{}gens\PYZus{}strs}\PY{p}{)}
\end{Verbatim}
\end{tcolorbox}

    \begin{Verbatim}[commandchars=\\\{\}]
logical operators X\_L=IIIIY and Z\_L=IIIIX work.
Pauli operators: lX, g1 commute.
Pauli operators: lX, g2 commute.
Pauli operators: lX, g3 commute.
Pauli operators: lX, g4 commute.
Pauli operators: lX, lX commute.
Pauli operators: lX, lZ anti-commute.
Pauli operators: lZ, g1 commute.
Pauli operators: lZ, g2 commute.
Pauli operators: lZ, g3 commute.
Pauli operators: lZ, g4 commute.
Pauli operators: lZ, lX anti-commute.
Pauli operators: lZ, lZ commute.
    \end{Verbatim}

    \begin{tcolorbox}[breakable, size=fbox, boxrule=1pt, pad at break*=1mm,colback=cellbackground, colframe=cellborder]
\prompt{In}{incolor}{33}{\boxspacing}
\begin{Verbatim}[commandchars=\\\{\}]
\PY{c+c1}{\PYZsh{} Test problem 2 code generators and logical ops }
\PY{c+c1}{\PYZsh{} !!! This is incredibly inefficient, i\PYZsq{}m so sorry if you actually run this !!!}
\PY{n}{prob2\PYZus{}code\PYZus{}strs} \PY{o}{=} \PY{p}{\PYZob{}}
    \PY{l+s+s2}{\PYZdq{}}\PY{l+s+s2}{g1}\PY{l+s+s2}{\PYZdq{}}\PY{p}{:} \PY{l+s+s2}{\PYZdq{}}\PY{l+s+s2}{ZIIIZII}\PY{l+s+s2}{\PYZdq{}}\PY{p}{,}
    \PY{l+s+s2}{\PYZdq{}}\PY{l+s+s2}{g2}\PY{l+s+s2}{\PYZdq{}}\PY{p}{:} \PY{l+s+s2}{\PYZdq{}}\PY{l+s+s2}{IZIIZII}\PY{l+s+s2}{\PYZdq{}}\PY{p}{,}
    \PY{l+s+s2}{\PYZdq{}}\PY{l+s+s2}{g3}\PY{l+s+s2}{\PYZdq{}}\PY{p}{:} \PY{l+s+s2}{\PYZdq{}}\PY{l+s+s2}{IIZIIZI}\PY{l+s+s2}{\PYZdq{}}\PY{p}{,}
    \PY{l+s+s2}{\PYZdq{}}\PY{l+s+s2}{g4}\PY{l+s+s2}{\PYZdq{}}\PY{p}{:} \PY{l+s+s2}{\PYZdq{}}\PY{l+s+s2}{IIIZIIZ}\PY{l+s+s2}{\PYZdq{}}\PY{p}{,}
    \PY{l+s+s2}{\PYZdq{}}\PY{l+s+s2}{g5}\PY{l+s+s2}{\PYZdq{}}\PY{p}{:} \PY{l+s+s2}{\PYZdq{}}\PY{l+s+s2}{IIXXIYY}\PY{l+s+s2}{\PYZdq{}}\PY{p}{,}
    \PY{l+s+s2}{\PYZdq{}}\PY{l+s+s2}{g6}\PY{l+s+s2}{\PYZdq{}}\PY{p}{:} \PY{l+s+s2}{\PYZdq{}}\PY{l+s+s2}{XXXZXXI}\PY{l+s+s2}{\PYZdq{}}\PY{p}{,}
\PY{p}{\PYZcb{}}

\PY{n}{find\PYZus{}logical\PYZus{}ops}\PY{p}{(}\PY{n}{prob2\PYZus{}code\PYZus{}strs}\PY{p}{)}
\end{Verbatim}
\end{tcolorbox}

    \begin{Verbatim}[commandchars=\\\{\}, frame=single, framerule=2mm, rulecolor=\color{outerrorbackground}]
\textcolor{ansi-red}{---------------------------------------------------------------------------}
\textcolor{ansi-red}{KeyboardInterrupt}                         Traceback (most recent call last)
\textcolor{ansi-green}{/var/folders/k8/wy10l4qx29x7j42dzjtj5zbh0000gn/T/ipykernel\_69985/1320410860.py} in \textcolor{ansi-cyan}{<module>}
\textcolor{ansi-green-intense}{\textbf{     10}} \}
\textcolor{ansi-green-intense}{\textbf{     11}} 
\textcolor{ansi-green}{---> 12}\textcolor{ansi-red}{ }find\_logical\_ops\textcolor{ansi-blue}{(}prob2\_code\_strs\textcolor{ansi-blue}{)}

\textcolor{ansi-green}{/var/folders/k8/wy10l4qx29x7j42dzjtj5zbh0000gn/T/ipykernel\_69985/487974432.py} in \textcolor{ansi-cyan}{find\_logical\_ops}\textcolor{ansi-blue}{(generators\_strs)}
\textcolor{ansi-green-intense}{\textbf{     25}}     \textcolor{ansi-green}{for} lZ \textcolor{ansi-green}{in} possible\_operators\textcolor{ansi-blue}{:}
\textcolor{ansi-green-intense}{\textbf{     26}}         \textcolor{ansi-green}{for} lX \textcolor{ansi-green}{in} possible\_operators\textcolor{ansi-blue}{:}
\textcolor{ansi-green}{---> 27}\textcolor{ansi-red}{             }lX\_op \textcolor{ansi-blue}{=} create\_operator\_from\_string\textcolor{ansi-blue}{(}lX\textcolor{ansi-blue}{)}
\textcolor{ansi-green-intense}{\textbf{     28}}             lZ\_op \textcolor{ansi-blue}{=} create\_operator\_from\_string\textcolor{ansi-blue}{(}lZ\textcolor{ansi-blue}{)}
\textcolor{ansi-green-intense}{\textbf{     29}}             \textcolor{ansi-green}{if}\textcolor{ansi-blue}{(}np\textcolor{ansi-blue}{.}array\_equal\textcolor{ansi-blue}{(}find\_anticommutator\textcolor{ansi-blue}{(}lX\_op\textcolor{ansi-blue}{,}lZ\_op\textcolor{ansi-blue}{)}\textcolor{ansi-blue}{,}ZERO\textcolor{ansi-blue}{)}\textcolor{ansi-blue}{)}\textcolor{ansi-blue}{:}

\textcolor{ansi-green}{/var/folders/k8/wy10l4qx29x7j42dzjtj5zbh0000gn/T/ipykernel\_69985/2932398810.py} in \textcolor{ansi-cyan}{create\_operator\_from\_string}\textcolor{ansi-blue}{(op\_string)}
\textcolor{ansi-green-intense}{\textbf{      8}}     op \textcolor{ansi-blue}{=} I
\textcolor{ansi-green-intense}{\textbf{      9}}     \textcolor{ansi-green}{for} c \textcolor{ansi-green}{in} op\_string\textcolor{ansi-blue}{:}
\textcolor{ansi-green}{---> 10}\textcolor{ansi-red}{         }op \textcolor{ansi-blue}{=} np\textcolor{ansi-blue}{.}matmul\textcolor{ansi-blue}{(}op\textcolor{ansi-blue}{,} pauli\_group\textcolor{ansi-blue}{[}c\textcolor{ansi-blue}{]}\textcolor{ansi-blue}{)}
\textcolor{ansi-green-intense}{\textbf{     11}}     \textcolor{ansi-green}{return} op
\textcolor{ansi-green-intense}{\textbf{     12}} 

\textcolor{ansi-red}{KeyboardInterrupt}: 
    \end{Verbatim}

    I was not able to get this function to get the logical X and Z operators
in a reasonable amount of time.

    \hypertarget{show-that-transforming-the-stabilizer-generators-by-applying-a-single-qubit-clifford-gate-to-any-qubit-still-yields-a-code-that-corrects-all-single-qubit-errors.}{%
\subsection{3) Show that transforming the stabilizer generators by
applying a single qubit clifford gate to any qubit still yields a code
that corrects all single qubit
errors.}\label{show-that-transforming-the-stabilizer-generators-by-applying-a-single-qubit-clifford-gate-to-any-qubit-still-yields-a-code-that-corrects-all-single-qubit-errors.}}

    \begin{tcolorbox}[breakable, size=fbox, boxrule=1pt, pad at break*=1mm,colback=cellbackground, colframe=cellborder]
\prompt{In}{incolor}{22}{\boxspacing}
\begin{Verbatim}[commandchars=\\\{\}]
\PY{c+c1}{\PYZsh{} Apply a H gate to each Pauli and add new operators to Pauli group}
\PY{n}{H}\PY{o}{=} \PY{n}{cmath}\PY{o}{.}\PY{n}{sqrt}\PY{p}{(}\PY{l+m+mi}{1}\PY{o}{/}\PY{l+m+mi}{2}\PY{p}{)}\PY{o}{*}\PY{p}{(}\PY{n}{X}\PY{o}{+}\PY{n}{Z}\PY{p}{)}
\PY{n}{pauli\PYZus{}group}\PY{o}{.}\PY{n}{update}\PY{p}{(}\PY{p}{\PYZob{}}
    \PY{l+s+s2}{\PYZdq{}}\PY{l+s+s2}{i}\PY{l+s+s2}{\PYZdq{}}\PY{p}{:} \PY{n}{np}\PY{o}{.}\PY{n}{matmul}\PY{p}{(}\PY{n}{H}\PY{p}{,}\PY{n}{I}\PY{p}{)}\PY{p}{,}
    \PY{l+s+s2}{\PYZdq{}}\PY{l+s+s2}{x}\PY{l+s+s2}{\PYZdq{}}\PY{p}{:} \PY{n}{np}\PY{o}{.}\PY{n}{matmul}\PY{p}{(}\PY{n}{H}\PY{p}{,}\PY{n}{X}\PY{p}{)}\PY{p}{,}
    \PY{l+s+s2}{\PYZdq{}}\PY{l+s+s2}{y}\PY{l+s+s2}{\PYZdq{}}\PY{p}{:} \PY{n}{np}\PY{o}{.}\PY{n}{matmul}\PY{p}{(}\PY{n}{H}\PY{p}{,}\PY{n}{Y}\PY{p}{)}\PY{p}{,}
    \PY{l+s+s2}{\PYZdq{}}\PY{l+s+s2}{z}\PY{l+s+s2}{\PYZdq{}}\PY{p}{:} \PY{n}{np}\PY{o}{.}\PY{n}{matmul}\PY{p}{(}\PY{n}{H}\PY{p}{,}\PY{n}{Z}\PY{p}{)}
\PY{p}{\PYZcb{}}\PY{p}{)}

\PY{c+c1}{\PYZsh{} Modify generators to have H applied to first qubit}
\PY{n}{prob2\PYZus{}code\PYZus{}strs} \PY{o}{=} \PY{p}{\PYZob{}}
    \PY{l+s+s2}{\PYZdq{}}\PY{l+s+s2}{g1}\PY{l+s+s2}{\PYZdq{}}\PY{p}{:} \PY{l+s+s2}{\PYZdq{}}\PY{l+s+s2}{zIIIZII}\PY{l+s+s2}{\PYZdq{}}\PY{p}{,}
    \PY{l+s+s2}{\PYZdq{}}\PY{l+s+s2}{g2}\PY{l+s+s2}{\PYZdq{}}\PY{p}{:} \PY{l+s+s2}{\PYZdq{}}\PY{l+s+s2}{iZIIZII}\PY{l+s+s2}{\PYZdq{}}\PY{p}{,}
    \PY{l+s+s2}{\PYZdq{}}\PY{l+s+s2}{g3}\PY{l+s+s2}{\PYZdq{}}\PY{p}{:} \PY{l+s+s2}{\PYZdq{}}\PY{l+s+s2}{iIZIIZI}\PY{l+s+s2}{\PYZdq{}}\PY{p}{,}
    \PY{l+s+s2}{\PYZdq{}}\PY{l+s+s2}{g4}\PY{l+s+s2}{\PYZdq{}}\PY{p}{:} \PY{l+s+s2}{\PYZdq{}}\PY{l+s+s2}{iIIZIIZ}\PY{l+s+s2}{\PYZdq{}}\PY{p}{,}
    \PY{l+s+s2}{\PYZdq{}}\PY{l+s+s2}{g5}\PY{l+s+s2}{\PYZdq{}}\PY{p}{:} \PY{l+s+s2}{\PYZdq{}}\PY{l+s+s2}{iIXXIYY}\PY{l+s+s2}{\PYZdq{}}\PY{p}{,}
    \PY{l+s+s2}{\PYZdq{}}\PY{l+s+s2}{g6}\PY{l+s+s2}{\PYZdq{}}\PY{p}{:} \PY{l+s+s2}{\PYZdq{}}\PY{l+s+s2}{xXXZXXI}\PY{l+s+s2}{\PYZdq{}}\PY{p}{,}
\PY{p}{\PYZcb{}}

\PY{n}{prob2\PYZus{}code\PYZus{}generated} \PY{o}{=} \PY{n}{create\PYZus{}operator\PYZus{}matrix}\PY{p}{(}\PY{n}{prob2\PYZus{}code\PYZus{}strs}\PY{p}{)}

\PY{k}{for} \PY{p}{(}\PY{n}{kop1}\PY{p}{,} \PY{n}{op1}\PY{p}{)} \PY{o+ow}{in} \PY{n}{prob2\PYZus{}code\PYZus{}generated}\PY{o}{.}\PY{n}{items}\PY{p}{(}\PY{p}{)}\PY{p}{:}
    \PY{k}{for} \PY{p}{(}\PY{n}{kop2}\PY{p}{,} \PY{n}{op2}\PY{p}{)} \PY{o+ow}{in} \PY{n}{prob2\PYZus{}code\PYZus{}generated}\PY{o}{.}\PY{n}{items}\PY{p}{(}\PY{p}{)}\PY{p}{:}
        \PY{n}{commutator} \PY{o}{=} \PY{n}{find\PYZus{}commutator}\PY{p}{(}\PY{n}{op1}\PY{p}{,}\PY{n}{op2}\PY{p}{)}
        \PY{n}{anticommutator} \PY{o}{=} \PY{n}{find\PYZus{}anticommutator}\PY{p}{(}\PY{n}{op1}\PY{p}{,}\PY{n}{op2}\PY{p}{)}
        \PY{k}{if} \PY{n}{np}\PY{o}{.}\PY{n}{array\PYZus{}equal}\PY{p}{(}\PY{n}{commutator}\PY{p}{,}\PY{n}{ZERO}\PY{p}{)} \PY{o+ow}{is} \PY{k+kc}{True}\PY{p}{:} 
            \PY{n+nb}{print}\PY{p}{(}\PY{l+s+sa}{f}\PY{l+s+s2}{\PYZdq{}}\PY{l+s+s2}{Pauli operators: }\PY{l+s+si}{\PYZob{}}\PY{n}{kop1}\PY{l+s+si}{\PYZcb{}}\PY{l+s+s2}{, }\PY{l+s+si}{\PYZob{}}\PY{n}{kop2}\PY{l+s+si}{\PYZcb{}}\PY{l+s+s2}{ commute.}\PY{l+s+s2}{\PYZdq{}}\PY{p}{)}
        \PY{k}{if} \PY{n}{np}\PY{o}{.}\PY{n}{array\PYZus{}equal}\PY{p}{(}\PY{n}{anticommutator}\PY{p}{,}\PY{n}{ZERO}\PY{p}{)} \PY{o+ow}{is} \PY{k+kc}{True}\PY{p}{:} 
            \PY{n+nb}{print}\PY{p}{(}\PY{l+s+sa}{f}\PY{l+s+s2}{\PYZdq{}}\PY{l+s+s2}{Pauli operators: }\PY{l+s+si}{\PYZob{}}\PY{n}{kop1}\PY{l+s+si}{\PYZcb{}}\PY{l+s+s2}{, }\PY{l+s+si}{\PYZob{}}\PY{n}{kop2}\PY{l+s+si}{\PYZcb{}}\PY{l+s+s2}{ anti\PYZhy{}commute.}\PY{l+s+s2}{\PYZdq{}}\PY{p}{)}
\end{Verbatim}
\end{tcolorbox}

    \begin{Verbatim}[commandchars=\\\{\}]
Pauli operators: g1, g1 commute.
Pauli operators: g1, g2 commute.
Pauli operators: g1, g3 commute.
Pauli operators: g1, g4 commute.
Pauli operators: g1, g5 commute.
Pauli operators: g2, g1 commute.
Pauli operators: g2, g2 commute.
Pauli operators: g2, g3 commute.
Pauli operators: g2, g4 commute.
Pauli operators: g2, g5 commute.
Pauli operators: g3, g1 commute.
Pauli operators: g3, g2 commute.
Pauli operators: g3, g3 commute.
Pauli operators: g3, g4 commute.
Pauli operators: g3, g5 commute.
Pauli operators: g4, g1 commute.
Pauli operators: g4, g2 commute.
Pauli operators: g4, g3 commute.
Pauli operators: g4, g4 commute.
Pauli operators: g4, g5 commute.
Pauli operators: g5, g1 commute.
Pauli operators: g5, g2 commute.
Pauli operators: g5, g3 commute.
Pauli operators: g5, g4 commute.
Pauli operators: g5, g5 commute.
Pauli operators: g6, g6 commute.
    \end{Verbatim}

    Since each generator still commutes with each other generator with an H
gate applied to the first qubit, we know that we can still generate any
correction that we were able to before with the regular generators.
Therefore, we are still able to correct any single qubit error.


    % Add a bibliography block to the postdoc
    
    
    
\end{document}
